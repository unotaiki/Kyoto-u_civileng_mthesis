\crefname{equation}{式}{式}% {環境名}{単数形}{複数形} \crefで引くときの表示
\crefname{figure}{図}{図}% {環境名}{単数形}{複数形} \crefで引くときの表示
\crefname{table}{表}{表}% {環境名}{単数形}{複数形} \crefで引くときの表示
\crefname{algorithm}{Algorithm}{Algorithm}

\crefname{section}{第}{第}
\creflabelformat{section}{#2#1節#3}
\crefname{subsection}{第}{第}
\creflabelformat{subsection}{#2#1小節#3}

\theoremstyle{definition}% 日本語用.定理とか斜体にならないようにする
\newtheorem{theorem}{定理}
\crefname{theorem}{定理}{定理}
\newtheorem{lemma}{補題}
\crefname{lemma}{補題}{補題}
\newtheorem{corollary}{系}
\crefname{corollary}{系}{系}
\newtheorem*{proof*}{証明}
\crefname{proof*}{証明}{証明}
\newtheorem{assumption}{仮定}
\crefname{assumption}{仮定}{仮定}
\newtheorem{definition}{定義}
\crefname{definition}{定義}{定義}
\newtheorem{remark}{注意}
\crefname{remark}{注意}{注意}
\newtheorem{proposition}{命題}
\crefname{proposition}{命題}{命題}


\newcommand{\crefpairconjunction}{と}
\newcommand{\crefrangeconjunction}{から}
\newcommand{\crefmiddleconjunction}{,}
\newcommand{\creflastconjunction}{,および}