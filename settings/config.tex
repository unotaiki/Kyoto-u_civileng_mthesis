%===============================================================================
% This file is based on the following source:
%
% ファイル名: master_ja_sample.tex
% 作成者:サイトウ リョウタ(2024年度山上研究室 修士修了)
% 作成日:2025年2月19日
% https://github.com/ryota-saito-coastal/LaTeXsample_for_Kyoto-u_civileng_mthesis
%===============================================================================

% --- 基本パッケージ ---
\usepackage{luatexja}
\usepackage{luatexja-fontspec}
\usepackage[haranoaji]{luatexja-preset}

\usepackage{amsmath,amssymb,amsthm}% 数式関連のパッケージ
\usepackage{mathtools}
\mathtoolsset{showonlyrefs=true}
\usepackage[unicode=true,hypertexnames=false,setpagesize=false,colorlinks=true,urlcolor=blue,linkcolor=blue,citecolor=blue]{hyperref}
\usepackage{cleveref}
\crefname{equation}{式}{式}% {環境名}{単数形}{複数形} \crefで引くときの表示
\crefname{figure}{図}{図}% {環境名}{単数形}{複数形} \crefで引くときの表示
\crefname{table}{表}{表}% {環境名}{単数形}{複数形} \crefで引くときの表示
\crefname{algorithm}{Algorithm}{Algorithm}

\crefname{section}{第}{第}
\creflabelformat{section}{#2#1節#3}
\crefname{subsection}{第}{第}
\creflabelformat{subsection}{#2#1小節#3}

\theoremstyle{definition}% 日本語用.定理とか斜体にならないようにする
\newtheorem{theorem}{定理}
\crefname{theorem}{定理}{定理}
\newtheorem{lemma}{補題}
\crefname{lemma}{補題}{補題}
\newtheorem{corollary}{系}
\crefname{corollary}{系}{系}
\newtheorem*{proof*}{証明}
\crefname{proof*}{証明}{証明}
\newtheorem{assumption}{仮定}
\crefname{assumption}{仮定}{仮定}
\newtheorem{definition}{定義}
\crefname{definition}{定義}{定義}
\newtheorem{remark}{注意}
\crefname{remark}{注意}{注意}
\newtheorem{proposition}{命題}
\crefname{proposition}{命題}{命題}


\newcommand{\crefpairconjunction}{と}
\newcommand{\crefrangeconjunction}{から}
\newcommand{\crefmiddleconjunction}{,}
\newcommand{\creflastconjunction}{,および} 
\usepackage{mathtools}

\usepackage{physics}
\usepackage[ppl]{mathcomp}
\usepackage{bm}
\usepackage{siunitx}
\usepackage{graphicx}
\usepackage{subcaption}
\usepackage{float}       % [H] オプション用
\usepackage{booktabs}    % 表の罫線(topruleなど)
\usepackage{multirow}
\usepackage{xcolor}
\usepackage{ifthen}
\usepackage{geometry}
\usepackage{setspace}
\usepackage{titlesec}
\usepackage{fancyhdr}
\usepackage{tocloft}
\usepackage{indentfirst} % 章の最初の段落も字下げ
\usepackage{bookmark}
\usepackage{caption}
\usepackage{enumitem}
\usepackage{url}

% --- 参考文献設定 (Biber + BibLaTeX) ---
% 従来の \usepackage{cite} や thebibliography環境 の代わり
\usepackage[backend=biber, style=numeric, sorting=none]{biblatex}

% % 引用番号のスタイルを [1] から 1) に変更
% \DeclareFieldFormat{labelnumberwidth}{#1)}

% % 文中の引用 \cite{} を上付き文字にする設定
% \DeclareCiteCommand{\cite}[\mkbibsuperscript]
%   {\usebibmacro{cite:init}%
%    \usebibmacro{prenote}}
%   {\usebibmacro{citeindex}%
%    \printfield{labelnumber}\printtext{)}} % 閉じ括弧 ) を追加
%   {\multicitedelim}
%   {\usebibmacro{postnote}}

% 複数の引用をカンマ区切りにする
\renewcommand*{\multicitedelim}{\textsuperscript{,}}

% --- フォント設定 (LuaLaTeX用) ---
% 英数字フォント
\setmainfont{Times New Roman}[RawFeature={+palt, +kern}, LetterSpace=2.0]
\setsansfont{Arial}[RawFeature={+palt, +kern}, LetterSpace=1.5]
\setmonofont{Courier New}[RawFeature={+palt, +kern}, LetterSpace=1.0]

% 日本語フォント(必要に応じて有効化してください)
% \setmainjfont{MS Mincho}[RawFeature={+palt, +kern}, LetterSpace=12.5]
% \setsansjfont{MS Gothic}[RawFeature={+palt, +kern}, LetterSpace=10.0]
% \renewcommand{\bfseries}{\gtfamily}

% --- ページレイアウト ---
\geometry{left=20mm, right=20mm, top=25mm, bottom=25mm}
\setstretch{1.2}
\linespread{2.5} % 普通は1.25くらいらしい

% --- キャプション等の日本語化 ---
\renewcommand{\figurename}{\normalfont\gtfamily 図}
\renewcommand{\tablename}{\normalfont\gtfamily 表}
\renewcommand{\contentsname}{\normalfont\gtfamily 目次}
\renewcommand{\thefigure}{\thechapter-\arabic{figure}}
\renewcommand{\thetable}{\thechapter-\arabic{table}}

% --- 見出しデザイン (titlesec) ---
\titleformat{\chapter}
    {\normalfont\fontsize{20pt}{24pt}\gtfamily}
    {第\hspace{2pt}\thechapter 章}
    {20pt}
    {\fontsize{20pt}{10pt}\selectfont}

\titleformat{\section}{\normalfont\fontsize{12pt}{12pt}\gtfamily}{\thesection}{1em}{}
\titleformat{\subsection}{\normalfont\fontsize{12pt}{12pt}\gtfamily}{\thesubsection}{1em}{}

% --- 目次設定 (tocloft) ---
\renewcommand{\cftchappresnum}{第}
\renewcommand{\cftchapaftersnum}{章\hspace{0.5em}}
\setlength{\cftchapnumwidth}{4.0em}

% --- ヘッダー・フッター ---
\pagestyle{fancy}
\fancyhf{}
\fancyfoot[C]{\thepage}
\renewcommand{\headrulewidth}{0pt}

% --- 便利コマンド ---
\newcommand{\HRule}[1]{\rule{1\linewidth}{#1}}
\newcommand{\percent}{\%}
\newcommand{\figureref}[1]{図\ref{#1}}
\newcommand{\tableref}[1]{表\ref{#1}}

% --- デバッグモード設定 (目に優しい黒背景) ---
% \submissionmode は main.tex で定義する
\newcommand{\debugnote}{
    \ifthenelse{\submissionmode=0}{
        \textbf{\textcolor{red}{\LARGE{REMOVE THIS COLOR SETTING FOR SUBMISSION!}}} \\
    }{}
}
% 設定反映ロジック
\AtBeginDocument{
    \ifthenelse{\submissionmode=0}{
        \pagecolor{black}
        \color{white}
    }{}
}