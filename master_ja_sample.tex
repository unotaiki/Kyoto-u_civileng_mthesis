
\documentclass[a4paper, 11pt]{report}
\usepackage{luatexja}

% Required packages
\usepackage{amsmath, amsthm, amssymb, amsfonts}
\usepackage{graphicx}
\usepackage{physics}
\usepackage{subcaption}
\usepackage{geometry}
\usepackage{setspace}
\usepackage{hyperref}
\usepackage{titlesec}
\usepackage{fancyhdr}
\usepackage{bookmark}
\usepackage{caption}
\usepackage{enumitem}
\usepackage{siunitx}
\usepackage{luatexja-fontspec}
\usepackage{tocloft}
\usepackage{indentfirst} % Enable indentation for the first paragraph after a chapter or section
\usepackage{cite} % ref
\usepackage{float} % for \begin{figure}[H] option in figures, fix in the position
\usepackage{xcolor}
\usepackage{ifthen}
\usepackage{bm}
\usepackage{booktabs}
\usepackage[ppl]{mathcomp}
\usepackage{multirow}

\makeatletter
\renewcommand{\@cite}[2]{\textsuperscript{#1)}} % "1)"

\newcommand{\supercite}[2]{\cite{#1}\textsuperscript{,}\cite{#2}}

% Reference style
\renewenvironment{thebibliography}[1]
{\section*{\fontsize{18pt}{22pt}\selectfont 参考文献}
    \list{\arabic{enumi})}
        {\settowidth\labelwidth{#1)}%
        \leftmargin\labelwidth
        \advance\leftmargin\labelsep
        \usecounter{enumi}%
        \let\p@enumi\@empty
        \renewcommand\theenumi{\arabic{enumi})}}%
    \sloppy\clubpenalty4000\widowpenalty4000\sfcode`\.\@m}
{\endlist}
\urlstyle{same}   

% Set font to MS Mincho # only work in Windows
% \setmainjfont{MS Mincho}[
%     RawFeature={+palt, +kern},
%     LetterSpace=12.5 % Increase spacing between characters
% ]
% \setsansjfont{MS Gothic}[
%     RawFeature={+palt, +kern},
%     LetterSpace=10.0
% ]
% \renewcommand{\bfseries}{\gtfamily}

% Set font to Times New Roman for English
\setmainfont{Times New Roman}[
    RawFeature={+palt, +kern},
    LetterSpace=2.0
]
\setsansfont{Arial}[
    RawFeature={+palt, +kern},
    LetterSpace=1.5
]
\setmonofont{Courier New}[
    RawFeature={+palt, +kern},
    LetterSpace=1.0
]

% Page margins settings
\geometry{
    left=20mm,      % Left margin
    right=20mm,     % Right margin
    top=25mm,       % Top margin
    bottom=25mm     % Bottom margin
}

% Line spacing
\setstretch{1.2}
\renewcommand{\baselinestretch}{1.2}
\linespread{2.5}
\newcommand{\HRule}[1]{\rule{1\linewidth}{#1}}

% Rename figure and table captions to Japanese
\renewcommand{\figurename}{\normalfont\gtfamily 図} % Figure label to Japanese
\renewcommand{\tablename}{\normalfont\gtfamily 表} % Table label to Japanese
\renewcommand{\thefigure}{\thechapter-\arabic{figure}}
\renewcommand{\thetable}{\thechapter-\arabic{table}}
\newcommand{\figureref}[1]{図\ref{#1}}
\newcommand{\tableref}[1]{表\ref{#1}}

% Page numbering
\pagestyle{fancy}
\fancyhf{}
\fancyfoot[C]{\thepage} % Page number centered in the footer
\renewcommand{\headrulewidth}{0pt}

% =====================================================
% REMOVE FOR SUBMISSION!!!
% Set color for my eyes
% Disable or delete the following settings to revert 
% the background and text colors to normal settings!!!
% 
% Conditional flag: Submission mode (1: Submission, 0: Debug mode)
\newcommand{\submissionmode}{1} % Set to 1 for submission
% 
% =====================================================

% NOTE: REMOVE FOR SUBMISSION!
\newcommand{\debugnote}{
    \ifthenelse{\submissionmode=0}{
        \textbf{\textcolor{red}{\LARGE{REMOVE THIS COLOR SETTING FOR SUBMISSION!}}} \\
    }{}
}

\ifthenelse{\submissionmode=0}{
    \pagecolor{black}
    \color{white}
}{}

% =====================================================
% =====================================================
\titleformat{\chapter}
    {\normalfont\fontsize{20pt}{24pt}\gtfamily} % Style and font size
    {第\hspace{2pt}\thechapter 章} % Chapter format: "\u7b2c" ("第") and "\u7ae0" ("章") for "Chapter"
    {20pt} % Space between chapter number and title
    {\fontsize{20pt}{10pt}\selectfont} % Font size for title text

% Section and subsection title formatting
\titleformat{\section}
    {\normalfont\fontsize{12pt}{12pt}\gtfamily} % Style and font size
    {\thesection}{1em}{}

\titleformat{\subsection}
    {\normalfont\fontsize{12pt}{12pt}\gtfamily} % Style and font size
    {\thesubsection}{1em}{}

% Contents title to Japanese
\renewcommand{\contentsname}{\normalfont\gtfamily 目次} % Set "\u76ee\u6b21" ("目\u6b21") for "Contents"

% number of chaper, section, subsection
\renewcommand{\thechapter}{\arabic{chapter}}
\renewcommand{\thesection}{\arabic{chapter}.\arabic{section}}
\renewcommand{\thesubsection}{\thesection.\arabic{subsection}}

% tocloft
\renewcommand{\cftchappresnum}{第}
\renewcommand{\cftchapaftersnum}{章\hspace{0.5em}}

% for section and subsection
\renewcommand{\cftsecpresnum}{} 
\renewcommand{\cftsecaftersnum}{\hspace{0.5em}}
\renewcommand{\cftsubsecpresnum}{}
\renewcommand{\cftsubsecaftersnum}{\hspace{0.5em}}

% to output % (percent)
\newcommand{\percent}{\%}

% Adjust spacing in the table of contents
\setlength{\cftchapnumwidth}{4.0em} % Width for "第1章"

\begin{document}
\clearpage
\pagenumbering{roman} % i, ii, iii
\setlength{\baselineskip}{19.5pt}

% Title page
\thispagestyle{empty}

% Logo of Kyoto University
\noindent
\begin{minipage}[t]{\linewidth}
    \vspace{3.5cm}
    \raggedleft % Align to the right
    \raisebox{-1cm}[0pt][0pt]{
    \includegraphics[width=0.18\linewidth]{./img/title/kyodai.png}
    }
\end{minipage}

\vspace{-1.5cm}
\noindent
{\fontsize{12pt}{12pt}\gtfamily
\hspace{-0.3em}
\setlength{\baselineskip}{12pt}
\begin{minipage}[t]{\linewidth}
    京都大学大学院工学研究科 \\[-1.3em]
    社会基盤工学専攻修士論文 \\[-1.3em]
    令和8年2月 \\[-1.3em]
\end{minipage}
}

\noindent
{\fontsize{12pt}{12pt}\gtfamily
\hspace{-0.3em}
\setlength{\baselineskip}{12pt}
\begin{minipage}[t]{\linewidth}
    Master's Thesis \\[-1.3em]
    Department of Civil and Earth Resources Engineering \\[-1.3em]
    Graduate School of Engineering \\[-1.3em]
    Kyoto University \\[-1.3em]
    February 2025 \\[-1.7em]
\end{minipage}
}
\HRule{1.0pt}
\begin{center}
    \vspace{2.5cm}
    \debugnote
    {\fontsize{18pt}{36pt}\selectfont 
    \LaTeX による修士論文作成フォーマット
    } \\
    \vspace{6.5cm}
    {\fontsize{16pt}{18pt}\selectfont 
    京都大学大学院 \hspace{1em}工学研究科 \hspace{1em}社会基盤工学専攻 \\ 
    アイドル講座\hspace{1em}バーチャル音響学分野 \\ 
    星街\hspace{1em}すいせい \\
    }
\end{center}
\newpage

% Abstract
\begin{center}
    {\fontsize{12pt}{26pt}\selectfont \gtfamily 論文要旨} % 20pt font size, 24pt line spacing, gothic
\end{center}
\par \indent
ここから要旨を書き始めます.
「論文要旨」はセンタリングされます.

段落を変える際にはこのように空行を一行入れます.
「彗星のごとく現れたスターの原石!バーチャルアイドルの星街すいせいでーす!」
「彗星のごとく現れたスターの原石!バーチャルアイドルの星街すいせいでーす!」
「彗星のごとく現れたスターの原石!バーチャルアイドルの星街すいせいでーす!」
「彗星のごとく現れたスターの原石!バーチャルアイドルの星街すいせいでーす!」
「彗星のごとく現れたスターの原石!バーチャルアイドルの星街すいせいでーす!」
このテンプレートで推し活も捗ること間違いなし!推しのライブ配信が始まったら、とりあえずコンパイルボタンを押してから楽しみましょう。
このようにすれば段落は変わりません.
\newpage
% Table of contents
\tableofcontents
\newpage

\clearpage
\pagenumbering{arabic}

% Main content starts here
% ------------------------------------------------------------------------------
% ---chapter 1.-----------------------------------------------------------------
% ------------------------------------------------------------------------------
\chapter{はじめに}
この\LaTeX フォーマットは,京都大学大学院工学研究科社会基盤工学専攻・都市社会工学専攻の修士論文を執筆するための基本的な
機能付きのテキストファイルです.
とりあえず,2024年度の修士論文フォーマットには対応しています.

使用する\LaTeX コンパイラはLuaLaTeX環境を想定しています.
pLaTeX環境等をお使いの方は,適宜プリアンプルの修正を行ってください.

また,本フォーマットにはデバッグモードを実装しています.
出力されるpdfがブラック背景に白文字になるので,目の疲れを低減することができます.
submissionmodeの1を0に変更すればデバッグモードになります.

\section{新しい節}
節はsectionコマンドで入れます.

\subsection{新しい項}
項はsubsectionコマンドを用います.

% ------------------------------------------------------------------------------
% ---chapter 2.-----------------------------------------------------------------
% ------------------------------------------------------------------------------

\chapter{論文の執筆}
\section{図の挿入}
\begin{figure}[H]
    \centering
    \includegraphics[width=0.8\linewidth]{./img/chap2/computer_tokui_boy.png}
    \caption{コンピューターの得意な男の子}
    \label{fig_pcboy}
\end{figure}
[H]を使うと.texの入力通りの場所にfloatを配置できますが,文書が崩れる可能性があります.
基本的には[p]または[t]を用いるのがおすすめです.
このサンプルでは文章量の都合上[H]指定しています.
そのほかの設定やsubfigureの環境については調べてみてください.

\section{表の挿入}
\begin{table}[H]
    \centering
    \renewcommand{\arraystretch}{.8}
    \setlength{\tabcolsep}{12pt}
    \caption{星街すいせいのプロフィール\cite{11}}
    \begin{tabular}{c|c}
        \toprule
        \textbf{項目} & \textbf{内容} \\ 
        \midrule
        誕生日 & 3月22日 \\ 
        デビュー日 & 2018年3月22日 \\ 
        身長 & \SI{160}{cm} \\ 
        ユニット & ホロライブ0期生 \\ 
        メイク担当 & 手島nari \\ 
        ファンネーム & 星詠み \\ 
        配信タグ & \#ほしまちすたじお \\ 
        ファンアートタグ & \#ほしまちぎゃらりー \\ 
        \bottomrule
    \end{tabular}
    \label{tab_suisei_profile}
\end{table}
このように表を挿入することができます.
なぜか[H]を指定すると行間が大きくなってしまうので0.8を指定しましたが,[t]なら1.8程度でいいと思います.

\section{数式の挿入}
数式は文中への挿入またはequation環境での挿入の二種類に分かれます.
文中に挿入する際は,\(L = 4\pi R^2 \sigma T^4\)このように追加します.

独立表示させるには,
\begin{equation}
    r = \frac{a(1 - e^2)}{1 + e \cos\theta}
\end{equation}
このように書きます.

\section{引用・参照}
\tableref{tab_suisei_profile}において引用を行いましたが,引用はciteコマンドを用いています.
また,表や図,数式の参照を行う際はref系のコマンドで行います.

引用については,thebibliography環境を用いています.
自分が執筆した際は手動入力を行いました.
bibtex環境を用いてもいいと思いますが,コンパイルの手順や必要packageについてはご確認ください.

\section{その他}
SI単位系にはSIコマンドを用いています.
\SI{10}{m/s}等入力して,SI単位系付きの数値を適切に入力できます.

% ------------------------------------------------------------------------------
% ---chapter 3.-----------------------------------------------------------------
% ------------------------------------------------------------------------------
% chapter 3以降も同様に追加していきます.
% \chapter{以降の章タイトル}

% ------------------------------------------------------------------------------
% Acknowledgement
% ------------------------------------------------------------------------------

\newpage
\addtocontents{toc}{\vspace{3em}} % vspace in TOC
\section*{\fontsize{18pt}{22pt}\selectfont 謝辞} % Acknowledgement in TOC
\addcontentsline{toc}{section}{謝辞}
ここから謝辞が入ります.(基本的に主査の教授\rightarrow 副査の教授 \rightarrow 副査の准教授の先生の順,その後に
お世話になった先生や技術職員の方,他の学生等について言及する)
もちろん,推しアイドルの活躍を拝みながら書き上げたこともここに深く感謝申し上げます.

% ------------------------------------------------------------------------------
% Reference and Cited Works
% ------------------------------------------------------------------------------

\newpage
\addtocontents{toc}{\vspace{2em}} % vspace in TOC
\addcontentsline{toc}{section}{参考文献} % Reference in TOC
\begin{thebibliography}{99}
    % chapter 1
    \bibitem{11} 星街すいせい,ホロライブ公式サイト,\url{https://hololive.hololivepro.com/talents/hoshimachi-suisei/},(参照,2025-2-19).
    % chapter 2
    % chapterごとに分けておくと後で楽,pbibtexでのコンパイルを組み合わせてもいいと思います.
\end{thebibliography}

% ------------------------------------------------------------------------------

\end{document}

% ------------------------------------------------------------------------------
% ---end document---------------------------------------------------------------
% ------------------------------------------------------------------------------




