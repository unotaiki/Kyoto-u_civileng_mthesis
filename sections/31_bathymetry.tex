%!TEX root = ../main.tex
% ========================
% ==== 水深測量の諸手法 ===== 
% ========================
\section{水深測量の諸手法とその課題}\label{sec:bg_bathymetry-method-overview}
本節では、既存の水深測量技術を概観し、浅水域(Shallow Water)の三次元計測において生じる技術的課題を整理する。

\subsection{音響測深}\label{subsec:bg_sonar}


現在、水深測量(Bathymetry)の標準となっているのは、船舶に搭載したマルチビーム音響測深機(Multibeam Echosounder: MBES)である。
MBESは船底から扇状(Fan-shape)に音波を発射し、走査線上の多数の計測点の水深を同時に取得することで、面的な地形図を作成を可能とする。
深海域においては、一度の航行で数キロメートル幅の海底をスキャンできるため、Seabed 2030のような大規模な海洋マッピングプロジェクトの中核技術となっている。

しかし、浅水域においてMBESの計測効率は大きく低下する。
MBESの走査幅(Swath Width: $SW$)は、幾何学的に水深 $D$ と指向角 $\theta$(一般に$120^{\circ} \sim 150^{\circ}$)に依存して決定されるためである。
下式のようにその走査幅は水深に比例する:
\begin{equation}\label{eq:swath-width}
  SW = 2D \tan \left( \frac{\theta}{2} \right)
\end{equation}
水深 \qty{1000}{m} であれば \qty{3000}{m} 以上の幅を一度に計測可能であるが、日本の河川のような水深 \qty{1}{m} $\sim$ \qty{2}{m} の環境では、走査幅はわずか \qty{3}{m} $\sim$ \qty{8}{m} 程度に留まる。
したがって、対象水域を網羅するためには、探査船は数十回もの往復(測線)を繰り返す必要があり、時間的・金銭的コストの増大を招く。
また、送受波器の直下数十センチメートルは、音波の振動の余韻(リンギング)といった制約により、計測が不可能となる。
(例: Teledyne Marine社製 SeaBat T20-Pでは\cref{SeaBat_T20-P_product_leaflet} 深度50センチメートル以浅は、計測が不可能である)。
% トランスデューサーが「ドン!」と音波を発射した後、その振動(余韻)が完全に止まるまでにはわずかな時間がかかります(これをリンギングと呼びます)。振動している間にすぐ近くから反射波が帰ってきても、余韻にかき消されて受信できません。

加えて、物理的な制約も存在する。
浅水域においては、有人調査船は、座礁のリスクがあるため船体の喫水より浅い場所には侵入できない場合がある。
この課題に対し、近年では小型の無人水上艇(Unmanned Surface Vehicle: USV)を用いた測量も研究されている\cite{Giordano2015_sonar-bathymetry,Kurowski2019_survey-USV}。
USVは浅水域へのアクセスを可能にするが、流水環境下での姿勢制御や自己位置推定の難易度が高く、また走査幅の物理的制約(\cref{eq:swath-width})は浅水域ではより一層深刻になるため、広範囲を高頻度で計測するには依然としてコストが高い。
さらに、音響測深は水域のみに限定されるため、陸域から水域へ連続する地形を単独で計測することは不可能であり、UAV写真測量等とのデータ統合が別途必要となる。

\IncludeThreeImages[3cm]
  {figure/30_/sonar_1.jpg}{マルチビームソナーの概念図}
  {figure/30_/sonar_2.jpg}{マルチビームソナーによる水深測量図}
  {figure/30_/sonar_usv.png}{無人水上艇(USV)による水深測量例}
  {
  マルチビームソナーによる水深測量。
  \cite{Giordano2015_sonar-bathymetry}より引用。
  }
  {fig:sonar-figure}


\subsection{航空レーザ測深 (Airborne LiDAR Bathymetry)}\label{subsec:bg_alb}
航空レーザ測深(Airborne LiDAR Bathymetry: ALB)は、航空機から水を透過する緑色レーザ(波長\qty{532}{nm}付近)を照射し、水面反射と水底反射の時間差から水深を求める技術である\cite{Saylam2018_ALB}。
自らエネルギーを照射して観測を行うため、能動的リモートセンシング(Active Remote Sensing)に分類される。
濁度などの水質に大きく依存するが、最大で15 m程度まで計測可能であり、\cite{asia_survey_ALB}によると水深3m程度までを高密度に測定可能である。
日本国内においても、国土交通省により一級河川の定期横断測量への導入が進められており\cite{milt_river_3d_application_manual}、陸域と水域をシームレスに、かつ高密度に計測できる点が大きな利点である。

しかし、ALBには本研究が対象とする「浅水域の高頻度モニタリング」において、導入・運用コストの課題を抱えている。
有人航空機や大型ドローンに搭載する高出力LiDARは極めて高価(数千万円規模)である。
国家規模で行う5年に1度の定期測量には適しているが、出水のたびに地形変化を追跡するような機動的な運用を、各研究室規模で行うことは経済的に難しい。

\begin{figure}[htbp]
  \centering
  \includegraphics[width=0.80\textwidth]{figure/30_/Janga2023_passive_vs_active.png}
  \caption{
    能動的リモートセンシング(Active Remote Sensing)と受動的リモートセンシング(Passive Remote Sensing)の比較。
    \cite{Janga2023_RS-AI-Earth-Sciences}より引用。
    (a) 受動的リモートセンシング:センサが外部からの情報を受信する。
    (b) 能動的リモートセンシング:センサが自ら情報を発信し、その反射を受信する。
  }
  \label{fig:passive-vs-active-remote-sensing}
\end{figure}

\begin{figure}[htbp]
  \centering
  \includegraphics[width=0.60\textwidth]{figure/30_/ALB.png}
  \caption{
    航空レーザ測深(Airborne LiDAR Bathymetry: ALB)の原理。
    \cite{asia_survey_ALB}より引用。
    水面で反射する近赤外レーザと、水中を透過し水底で反射するグリーンレーザを用い、水底の三次元点群を得る。
    }
  \label{fig:passive-vs-active-remote-sensing}
\end{figure}


\missingfigure{各手法のPros \& Cons をTableにする。手法、測定可能深度、空間解像度、浅水域での効率、コスト、頻度、安全性、備考}



\subsection{Spectrally Derived Bathymetry}\label{subsec:bg_sdb} 
分光水深測量(Spectrally Derived Bathymetry: SDB)は、LandsatやSentinel-2などのマルチスペクトル衛星画像の分光特性を利用し、放射伝達モデルに基づいて水深を推定する手法である\cite{He2024_survey-shallow-bathymetry}。
\checkref{SDBに関して特化したSurvey論文があったように思える}
本手法は受動的リモートセンシング(Passive Remote Sensing)に分類され、広域かつ低コストに水深情報を取得できる利点がある一方で、その推定原理に起因する理論的・実用的な制約が存在する。

SDBの基本原理は、水中における光の指数関数的な減衰に基づく。
可視光領域において、水深が深いほど水底からの反射光は水柱による吸収・散乱を受けて減衰する。
SDBはこの物理現象を利用し、主に青~緑バンド(水中透過率が高い)と赤~近赤外バンド(水中減衰が大きい)の輝度比や、各バンドの反射率の対数線形モデル(Lyzenga法やStumpf法など)を用いて水深を回帰的に推定する。
\checkref{Lyzenga法やStumpf法などの引用}
SDBは海洋沿岸部等の広域な浅海域では有効な手法である一方、中小河川環境への適用においては、幾何学的アプローチである写真測量と比較して以下のような限界を有する。

\begin{itemize} 
  \item \textbf{放射量依存性と水質・底質の不均一性}: SDBは幾何学的な三次元復元ではなく、観測された放射輝度(Radiance)に基づく放射量的な推定(Radiometric estimation)である。
  そのため、推定精度は水域に固有の光学的特性(濁度やクロロフィルa濃度)や、水底の底質(砂、礫、植生)によるアルベドの空間的不均一性に強く依存する。
  
  \item \textbf{経験的モデルと現地データの必要性}: 一般的なバンド比法などの経験的モデルでは、輝度値を水深値へ変換するために現地での実測水深データを用いたキャリブレーション(回帰モデルの補正)が不可欠である。
  これは「計測なしで水深を得る」というリモートセンシングの利点を部分的に損なうものである。
  
  \item \textbf{空間解像度とミクセル問題}: Sentinel-2(\qty{10}{m})やLandsat(\qty{30}{m})などのオープンデータである衛星画像は、中小河川における三次元情報のニーズに対して解像度が不足する。
  1ピクセル内に水域、陸域、河畔林が混在する「ミクセル(Mixed Pixel)」問題が発生し、水深推定を阻害する。 
\end{itemize}


\begin{figure}[htbp]
  \centering
  \includegraphics[width=0.95\textwidth]{figure/30_/Kujawa2025_RS-bathymetry-methods.png}
  \caption{
    リモートセンシングによる水深測量技術の分類。
    \cite{Kujawa2025_survey-shallow-bathymetry}より引用。
    (a) 能動的・受動的手法の分類。(b) 各手法の空間分解能とカバレッジの関係。
  }
  \label{fig:remote-sensing-bathymetry-method}
\end{figure}


\subsection{空中写真測量 (Photogrammetric Bathymetry)}\label{subsec:bg_photogrammetric-bathymetry-overview}
以上の既存手法の課題を踏まえ、近年急速に普及したドローンなどの無人航空機(Unmanned Aerial Vehicle: UAV)用いた写真測量(Photogrammetry)は、低コスト・高解像度・高頻度なデータ取得が可能であることから、浅瀬測量技術においても有効であると考えられる。
UAVにより空撮された多視点画像から、詳細を後述するStructure-from-Motion (SfM) \cite{schoenberger2016_colmap} および Multi-View Stereo (MVS) \cite{Furukawa2010_PatchMVS,Furukawa2015_MVS} を用いて3次元形状を復元するアプローチ\cref{sec:prelim_3d-reconstruction-from-images}は、陸部においては既に広い用途で実用化されている\cite{Bemis2014_UAV-photogrammetry,Gomez2016_UAV-photogrammetry-disaster,Iglhaut2019_UAV-photogrammetry-forestry}。
この技術を水中に適用する場合、その手法は「空中からの水深写真測量(Photogrammetric Bathymetry)」と呼ばれ、多視点画像からの水中三次元再構成問題として定式化される。
これを実現するためには、光の反射や、水中での光の散乱・吸収による減衰、波による被写体の歪みなどの課題が存在するが、中でも最も根本的で重要な課題として、水面における「光の屈折(Refraction)」を克服しなければならない。
\missingfigure{図で屈折が既存の光の直進性やPinholeモデルを成立させなくことを示す}

既存のSfM/MVSアルゴリズムの大部分は、カメラと被写体が同一媒質(空気中)にあることを前提とし、光が直進するという幾何的な仮定に基づいている。
しかし、上空からの水中撮影においては、光は水中から空気中へ進む際に、媒質境界(水面)でスネルの法則(Snell's Law)に従って屈折する。
この屈折現象により、カメラから観測される被写体の「見かけの位置(Apparent Position)」は、実際の位置よりも浅く、かつ歪んで観測される。
したがって、従来の陸上用アルゴリズムをそのまま適用すると、水深が過小評価され、復元形状が破綻する。
本研究では、この屈折を物理的正確に考慮した(Refraction-aware)新たな三次元再構成手法を導入することで、水面下の高精度な計測を実現する。
