%!TEX root = ../main.tex
\chapter{序論}
\section{研究背景}
正確な水深情報(Bathymetry Information)の取得は、洪水予測~\cite{Grimaldi2018_flood-forecasting}、河床地形変動モニタリング~\cite{Hemmelder2018_monitoring-river-morphology}、水生生物生息環境評価~\cite{Thomson2001_habitat-assessment}、沿岸域管理~\cite{Pasquali2021_coastal-zone-management}などの様々な分野において重要である。
従来の船載ソナー~\cite{Giordano2015_sonar-bathymetry}は深海では有効であるが、浅海では座礁リスク、狭帯域カバレッジ、高コストな運用が課題となっている。
近年、航空機に搭載された光電測量システム(Airborne Light Detection and Ranging: ALB)が導入されているが、設備コストと運用コストが高いという課題がある。
一方、無人航空機(Unmanned Aerial Vehicle: UAV)を用いた写真測量(Photogrammetry)は、低コストで広範囲を効率的に調査する手法として注目されている。
しかし、UAVによる水域地形計測においては、水面で発生する光の屈折が、従来の写真測量における幾何学的仮定(共線性条件)を根本的に破綻させるという課題がある。
既存の手法は、屈折を完全に正確に物理的正確性を持ってモデル化しない経験的な補正や反復的な後処理に依存するか、あるいは説明可能性を欠くブラックボックス的な深層学習モデルを用いるものが多く、形状の幾何学的忠実性と外観の写実性を両立させることは困難であった。

\section{研究目的}
本研究では、この課題を解決するために、物理的に忠実な二媒質屈折モデルを再構成パイプラインに直接組み込んだ``Refraction-aware 3D Gaussian Splatting''を提案する。
本手法の核心的な貢献は、水中の真の位置にある3D Gaussian を、航空画像上の見かけの位置へと解析的にマッピングする微分可能なパラメータ変換の導入にある。これにより、標準的な3D Gaussian Splattingの柔軟なフレームワークを維持しつつ、屈折あり画像からの密な3次元形状と詳細なテクスチャ情報の復元を実現した。

\section{論文構成}
本論文の構成は以下の通りである。

\begin{itemize}
  \item 第2章:先行研究
  \item 第3章:理論背景
  \item 第4章:提案手法
  \item 第5章:評価実験
  \item 第6章:考察
  \item 第7章:結論
\end{itemize}