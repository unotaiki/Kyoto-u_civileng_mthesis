%!TEX root = ../main.tex
\chapter{提案手法}\label{chap:method}

\begin{figure}[htbp]
  \centering
  \includegraphics[width=1.0\textwidth]{figure/50_method/overview.png}
  \caption{
    屈折補正型Gaussian Splattingの概要。
    本手法では、3次元シーンをGaussianプリミティブの集合で表現し、各カメラ視点ごとに「見かけ空間」へと変換する。
    具体的には、位置・スケール・不透明度の各パラメータを屈折を考慮した微分可能変換によって補正し、水面における光の屈折現象を物理的にモデル化する。
    変換後のGaussianをレンダリングし、屈折による歪みを含む入力画像と比較することで、再構成誤差を逆伝播させて元のGaussianパラメータを最適化する。
    これにより、屈折の影響を排除した3次元モデルの復元と、フォトリアリスティックな新規視点合成を実現する。
    }
    \label{fig:method-overview}
\end{figure}




\section{3D Covariance $\Sigma^{3D}$ Correction (3D共分散の補正)}\label{sec:3Dsigma-correction}

% \begin{figure}[htbp]
%   \centering
%   \includegraphics[width=0.7\textwidth]{figure/50_method/}
%   \caption{

%     }
%     \label{fig:}
% \end{figure}


前節で述べた位置補正 $\bm{p} \to \bm{p'}$ は非線形な変換であり、カメラから見た「見かけの空間(Apparent space)」は、実空間に対して非一様に歪み、圧縮された空間となる(\figureref{fig:space_compression_by_view})。特に、入射角が大きい領域や深度が深い領域ほど、光路の屈折による空間圧縮効果は顕著となる。もし、Gaussianの中心位置のみを補正し、その形状(スケール$\bm{s}$および回転$\bm{R}$)を実空間の定義のまま描画に用いた場合、圧縮された見かけの空間に対してGaussianが相対的に肥大化して投影されることとなる。これは、レンダリング画像において不自然なボケ(Blurring)や、境界面の意図しない膨張といった視覚的アーティファクトを引き起こす要因となる(\figureref{fig:artifact-w/o-scale-correction}参照)。したがって、幾何学的に正確なレンダリングを行うためには、Gaussianの形状を決定する分散共分散行列(Covariance matrix)に対しても、空間の歪みに応じた補正が必要となる。ここで、各Gaussianプリミティブは十分に小さいと仮定し、その中心$\bm{p}$近傍における屈折変換を、Jacobian $J_{app}$(式\ref{eq:Jacobian})を用いた局所的な線形近似(Affine変換)として扱う。実空間におけるGaussianの3次元分散共分散行列を$\Sigma^{3D}$とすると、多変量正規分布の線形変換の性質に基づき、見かけの空間における分散共分散行列$\Sigma^{3D}_{app}$は次式で導出される。\begin{equation}\label{eq:covariance-correction}\Sigma^{3D}{app} = J{app} \Sigma^{3D} J_{app}^\top\end{equation}この$\Sigma^{3D}{app}$は、屈折によって歪んだ局所空間におけるGaussianの適切な形状と配向を表現している。最終的なレンダリングプロセスでは、通常の3DGSと同様に、この補正された共分散行列$\Sigma^{3D}{app}$に対してView Transformationおよび射影変換(EWA Splatting等)を適用し、画像平面上での2次元共分散行列を算出する。これにより、水面屈折特有の非線形な歪みを考慮しつつ、高品質かつアーティファクトの少ない水中シーンの再現が可能となる。