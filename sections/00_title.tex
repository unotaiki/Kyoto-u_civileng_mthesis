%!TEX root = ../main.tex
\thispagestyle{empty}
\setlength{\baselineskip}{19.5pt}

\noindent
\begin{minipage}[b]{0.78\linewidth} % 左側:テキストブロック(幅78%)
    \fontsize{12pt}{18pt}\selectfont \gtfamily % {文字サイズ}{行間} \gtfamily: ゴシック体
    京都大学大学院工学研究科\\
    社会基盤工学専攻修士論文\\
    令和8年2月\\[2.0ex]
    % 英文タイトル前にやや余白を挿入
    \normalfont % 英文はゴシック体→ローマン体(可読性向上)
    \fontsize{12pt}{18pt}\selectfont
    Master's Thesis\\
    Department of Civil and Earth Resources Engineering\\
    Graduate School of Engineering\\
    Kyoto University\\
    February 2026
\end{minipage}
\hfill % 左右の間のバネ
\begin{minipage}[b]{0.20\linewidth} % 右側:ロゴブロック(幅20%)
    \raggedleft 
    \includegraphics[width=\linewidth]{./figure/00_title/kyodai.png}
\end{minipage}
\vspace{0.5em}

\HRule{1.0pt}

% ===== Title & Author======
\begin{center}
    \vspace{2.5cm}
    \debugnote
    {\fontsize{18pt}{36pt}\gtfamily\bfseries\selectfont 
    水面屈折モデルを導入した\\Gaussian Splattingによる河川浅水域の三次元計測
    } \\[1.0ex]
    {\fontsize{18pt}{36pt}\bfseries\selectfont 
    Refraction-Aware Gaussian Splatting \\for Shallow Water Bathymetry from UAV Imagery
    } \\
    \vspace{6.5cm}
    {\fontsize{16pt}{18pt}\selectfont 
    京都大学大学院 \hspace{1em}工学研究科 \\
    社会基盤工学専攻 \hspace{1em}空間情報学講座\\ 
    宇野\hspace{1em}大輝 \\
    }
\end{center}
\newpage