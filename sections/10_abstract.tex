%!TEX root = ../main.tex
\begin{center}
  {\fontsize{12pt}{26pt}\selectfont \gtfamily 論文要旨} % 20pt font size, 24pt line spacing, gothic
\end{center}
\par \indent

\note{ACRSかISPRSの内容を翻訳しているだけ。最後にどこまでできたか踏まえて書く。内容も増えるはず。}
水中の3次元地形の計測は水深測量(Bathymetry)と呼ばれている。

浅水域においても、河床・海底の水深測量は、地形変動のモニタリング、ハザードシミュレーション、および水生生物の生息環境評価において極めて重要である。
近年、無人航空機(Unmanned Aerial Vehicle: UAV)を用いた写真測量(Photogrammetry)は、広範囲を効率的に調査する手法として注目されている。
しかし、空中から水底を撮影する際、水面で発生する光の屈折が、従来の写真測量における幾何学的仮定(共線性条件)を根本的に破綻させるという課題がある。
既存の手法は、屈折を完全に正確に物理的正確性を持ってモデル化しない経験的な補正や反復的な後処理に依存するか、あるいは説明可能性を欠くブラックボックス的な深層学習モデルを用いるものが多く、形状の幾何学的忠実性と外観の写実性を両立させることは困難であった。

本論文では、この課題を解決するために、物理的に忠実な二媒質屈折モデルを再構成パイプラインに直接組み込んだ``Refraction-aware 3D Gaussian Splatting''を提案する。
本手法の核心的な貢献は、水中の真の位置にある3D Gaussian を、航空画像上の見かけの位置へと解析的にマッピングする微分可能なパラメータ変換の導入にある。これにより、標準的な3D Gaussian Splattingの柔軟なフレームワークを維持しつつ、屈折あり画像からの密な3次元形状と詳細なテクスチャ情報の復元を実現した。

評価実験では、屈折以外の光学的要因を排除し厳密な検証を行うため、物理ベースのレイトレーシングにより生成された河床のシミュレーションデータセットを用いた。
その結果、水深10 mのスケールにおいて許容誤差10 cmとした場合の幾何学的F1スコアは96\%を達成した。
さらに、新規視点合成(Novel View Synthesis)においては、PSNR 25.9 dB、SSIM 0.93という詳細な屈折なし画像の推定を達成した。
提案手法により、平坦な水面条件下において、河川、湖沼、沿岸域の低コストかつ高頻度な3Dモニタリングが可能となり、水域リモートセンシング分野に新たな方法論的基盤を提供する。



