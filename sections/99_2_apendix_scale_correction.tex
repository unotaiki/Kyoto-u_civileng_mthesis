%!TEX root = ../main.tex

\subsection{Scale Correction}\label{sec:scale-correction}

ここで、述べるスケール補正(Scale Correction)は、\cref{sec:3Dsigma-correction}で述べた手法の前手法であり、3D共分散補正ほど正確な物理モデルに基づいた補正ではない。
しかし、実装の検証\cref{sec:eval_simulated}において用いたためここで記す。


変換 $\mathbf{p}\rightarrow\mathbf{p}'$ は非線形であるため、見かけ空間では座標系が歪む(\cref{fig:space_compression_by_view})。
補償的なスケーリングを行わないと、空間が不自然に圧縮され、Gaussian は表面上でぼやけて膨張する(\cref{fig:artifact-w/o-scale-correction})。



このアーティファクトを抑えるため、ヤコビアン $J_{\text{app}}$ が正準基底をどのように変換するかを調べ、局所的な異方歪みを定量化する。
例えば、変換後の $x$ 軸の基底ベクトルは $J_{\text{app}}$ の第1列として
\cref{eq:Jacobian} から次のように書ける:
\begin{equation}
  \bm{e}_x'=
  J_{\text{app}}\bm{e}_x=
    \begin{bmatrix}
        \partial x'/\partial x \\
        \partial y'/\partial x \\
        \partial z'/\partial x
    \end{bmatrix}
\end{equation}
このノルムを方向スケール因子として
\begin{equation}
  s_x=\|\bm{e}_x'\|
  % = \sqrt{\left(\frac{\partial x'}{\partial x}\right)^2 + \left(\frac{\partial y'}{\partial x}\right)^2 + \left(\frac{\partial z'}{\partial x}\right)^2}
\end{equation}
と定義する。同様に $s_y, s_z$ を他の軸について得る。

$\lvert\det J_{\text{app}}\rvert$ を体積の圧縮量として用いるのは自然に見える。
しかし、この近似は入射角が大きい領域で過補償となり、Gaussian が過度に小さくなることを確認した。
そこで、方向スケール因子の幾何平均を用いる:
\begin{equation}
  V=s_x\,s_y\,s_z
  \label{eq:volume-compression-ratio}
\end{equation}
さらに、等方的な補正係数を
\begin{equation}
  S = V^{1/3}
  \label{scale-correction-factor}
\end{equation}
と定める。最終的なスケールは
\begin{equation}
  \bm{s}' = S\,\bm{s}
  \label{eq:scale-correction-edge}
\end{equation}
で与える。この等方的な補正により、屈折による膨張アーティファクトを抑えつつ、学習の安定性を保ち、プリミティブが過度に膨張・収縮することを防げる。
