%!TEX root = ../main.tex
\section{Color Correction (色補正)}\label{sec:color-correction}

\note{Color Correctionというよりも、 `Radiometric Correction'という方が適切でニュアンスが伝わりやすいと思い変えた
}


\note{Blenderでは、どのRGB補正が使われ画像が出力されるか確認}

空気水面間の屈折境界面を通して3次元シーンを観測する場合、Gaussianの幾何学的な変動に加え、放射輝度(Radiance)の変化に伴う「見かけの暗化」が生じる。
これは、フレネル反射や水中での散乱・吸収減衰を無視したとしても、幾何光学的な光束$\Phi$の広がり(Etendue; エタンデュ)の変化に起因して発生する現象である。
Snellの法則(式\cref{eq:snell})およびその微分式\cref{eq:dtheta_r/dtheta_i}より、以下の関係式が導かれる。
\begin{equation}
  n^2 \cos \theta_i \sin \theta_i d \theta_i = n^2 \cos \theta_r \sin \theta_r d \theta_r
\end{equation}
ここで、微小面積要素$dA$および方位角の微小変化$d\psi$を考慮し、微小立体角$d\omega = \sin\theta d\theta d\psi$との関係を用いると、エタンデュ保存則\parencite{etendue}として知られる以下の関係が得られる。
\begin{equation}\label{eq:etendue_preservation}
  \cos\theta_i d\omega_i dA = n^2 \cos\theta_r d\omega_r dA
\end{equation}
エテンデュは光学系における保存量であり、境界面での損失がないと仮定した場合、入射光線と屈折光線の間で保存される。
したがって、水中(Radiance $L_w$)から空気中(Radiance $L_a$)へ光が進む際、光束の幾何学的変化に伴い、放射輝度は屈折率の二乗の逆数で減衰される:
\begin{align}\label{eq:radiance_between_air_and_water}
  L_a &= \frac{d^2 \Phi }{dA \cos\theta_i d\theta_i} \notag \\
      &= \frac{d^2 \Phi }{n^2 dA \cos\theta_r d\theta_r} \notag \\
      &= \frac{1}{n^2} L_w 
\end{align}
すなわち、空気中のカメラから観測される輝度$L_a$は、水中の真の輝度$L_w$と比較して$1/n^2$倍に減衰し、\cref{fig:refraction-makes-scene-darker}のようにシーン全体が暗く観測される。

\IncludeTwoImages[4cm]
  {figure/50_method/0090_gt_background.png}{}
  {figure/50_method/0090_refraction_background.png}{}
  {Blender Cyclesによるパストレーシングでレンダリングしたシーン例。
  (a) 屈折効果なし。(b) 水面の屈折を介して観測。(b)は(a)に比べて暗く観測されている。}
  {fig:refraction-makes-scene-darker}

我々の目的は、この物理的な減衰を補正し、本来の明るさにおける色情報をGaussianのパラメータとして学習することにある。
GSにおいて、シーンの放射輝度は最終的なRGBカラー値にベイクされている。
つまり、復元されるシーンの色情報は、撮像記録された画像群における色空間のまま、各Gaussianの色パラメータ$\bm{c}_k$として保持・レンダリングされる。
ここでは、色空間が一般的なsRGB色空間(Gamma補正済み)であると仮定しモデル化を行う。
Gamma値を$\gamma$(通常$\gamma=2.2$程度)、RGBから輝度への変換ベクトルを$\mathbf{I}$ (例 $\mathbf{I}^\top = (0.2126, 0.7152, 0.0722)$)とすると、ピクセル強度$\mathrm{intensity}(\bm{\Gamma})$は以下のようにベクトルの内積で表される:
\begin{equation}
  \mathrm{intensity}(\bm{\Gamma}) = \langle\mathbf{I},\bm{\Gamma}^\gamma\rangle
\end{equation}
ここで$\bm{\Gamma}$はGamma補正後のピクセル輝度(RGB)である。

物理的な放射輝度の関係式(式\ref{eq:radiance_between_air_and_water})を満たすためには、補正後の色$\bm{\Gamma}'$は以下の条件を満たす必要がある。
\begin{align}\label{eq:correc-color-intensity-condition}
  \mathrm{intensity}(\Gamma'(\bm{x})) &= \frac{1}{n^2}\langle\mathbf{I},\bm{\Gamma}^\gamma\rangle \notag \\
                                      &= \langle\mathbf{I}, \frac{1}{n^2} \bm{\Gamma}^\gamma\rangle  \notag \\
\end{align}
これにより、補正後のピクセル値$\bm{\Gamma}'$は以下のように表される:
\begin{align}\label{eq:correc-pixel-value}
  & \left(\bm{\Gamma}'\right)^\gamma = \frac{1}{n^2} \bm{\Gamma}^\gamma \notag \\
  & \bm{\Gamma}' = n^{-\frac{2}{\gamma}} \Gamma
\end{align}
すなわち、レンダリング方程式(式\ref{eq:3dgs-alpha-blending})において、各Gaussianの色$\bm{c}_k$に対し、次式で表される補正を適用することでエタンデュ保存則を満たすことができる。
具体的には、補正後の色$\bm{c}'_k$は
\begin{equation}\label{eq:correc-color-intensity}
  \bm{c}'_k = n^{-\frac{2}{\gamma}} \bm{c}_k
\end{equation}
で与えられる。この補正式をレンダリング方程式に代入すると、
\begin{align}\label{eq:correc-color-intensity-rendering}
  \bm{\Gamma}'(\bm{x}) &= \sum_{k=1}^{K} \bm{c}'_k \alpha^{\text{pixel}}_k \prod_{j=1}^{k-1} (1-\alpha^{\text{pixel}}_j) \notag \\
  &= \sum_{k=1}^{K} \left(n^{-\frac{2}{\gamma}}\bm{c}_k\right) \alpha^{\text{pixel}}_k \prod_{j=1}^{k-1} (1-\alpha^{\text{pixel}}_j) \notag \\
  &= n^{-\frac{2}{\gamma}} \sum_{k=1}^{K} \bm{c}_k \alpha^{\text{pixel}}_k \prod_{j=1}^{k-1} (1-\alpha^{\text{pixel}}_j) \notag \\
  &= n^{-\frac{2}{\gamma}} \bm{\Gamma}(\bm{x})
\end{align}
となり、\cref{eq:correc-pixel-value}で示された補正を満たすことで、エタンデュ保存則から導かれる屈折による画像の暗化を相殺し、水中シーン本来の色情報を復元可能である。
ただし、本稿ではフレネル反射や水中における散乱・吸収減衰の効果は考慮していない。これらの詳細な光学効果については今後の課題とする。

