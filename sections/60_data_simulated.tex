%!TEX root = ../main.tex
\chapter{Dataset}\label{chap:dataset}
\section{Simulated Dataset}\label{sec:simulated-dataset}

\begin{figure}[htbp]
  \centering
  \includegraphics[width=0.70\textwidth]{figure/60_data/dataset-capturing.png}
  \caption{
    データ取得環境の概要。
    直下視(Nadir)に加え、斜視(Oblique View)の2種類の起動から対象の地点を撮影した。
    }
    \label{fig:dataset-capture}
\end{figure}

\note{
  * 実際の取得画像を示すべき
  * 新たなBlenderデータセットを完成させ、10種類ほどの多様なデータセットで複合的な検証を行いたかったが、、
}

本節では、本実験で使用したシミュレーションデータセットの詳細について述べる。
データ取得環境の概要を \cref{fig:dataset-capture} に示す。
実験環境の構築にあたり、本研究の主目的である「屈折による幾何学的歪み」の影響を厳密に検証するため、3DCGソフトウェアBlender\cite{blender}を用いたシミュレーション環境を採用した。
これにより、実環境の実画像において不可避である水面反射や水中での光の減衰・散乱の影響を排除し、純粋な屈折現象のみに焦点を当てた評価を可能とした。
河床のシーン構築は、礫のテクスチャに対してDisplacementモディファイアを適用することで、不規則な凹凸を持つ微細な地形メッシュを作成した。
レンダリングにはパストレーシングエンジンであるBlender Cyclesを使用し、物理的に忠実な画像を生成した。
水面モデルには歪みのない理想的な平面を仮定した。
画像の撮影条件は、Gaussian Splatting等の多視点幾何推定手法に有利な高密度な視差情報を確保するため、オーバーラップ率およびサイドラップ率を共に85\%と高く設定した。
撮影軌道は、河床中心を直下視(Nadir)するグリッド状のパスに加え、天頂角20°および40°で対象を取り囲むように円軌道から撮影する条件を採用した(Oblique View)。
地形の平均水深は10 mであり、水面からのカメラ高度も10 mと、深度に対する相対高度が1:1となる非常に屈折の盈虚を受けやすい撮影条件である。
このような多角的な観測幾何は、三次元復元における幾何学的制約を強める一方で、\cref{fig:space_compression_by_view} に示すように、浅い角度での観測に伴う大きな屈折歪みや見かけ位置のズレを導入することにも繋がる。
センサモデルには歪みのないピンホールカメラモデルを採用し、Field of View(FOV)は70°、画像解像度は$800 \times 800$ピクセルとした。データセットの総枚数は、新規視点合成の評価用画像を除き、計90枚である。

\missingfigure{取得画像例}