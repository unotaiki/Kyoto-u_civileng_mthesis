%!TEX root = ../main.tex
\section{水深写真測量の諸手法}

\url{https://www.notion.so/Photommetric-Bathymetry-2e002f0efa178010b4f3f365fdea4f29}


以下の\cref{tab:refraction_methods_comparison}に空中からの水深写真測量(Photogrammetric Bathymetry)における屈折補正手法を比較する。
一部は\cite{Kujawa2025_survey-shallow-bathymetry}を参考にしている。

\newpage

\begingroup
\renewcommand{\arraystretch}{1.3} % 行間調整
\small % 文字サイズを少し小さくして収まりやすくする

% {table}環境は使いません。直接 xltabular を書きます。
\begin{xltabular}{\textwidth}{
    @{} l l >{\RaggedRight\arraybackslash}X >{\RaggedRight\arraybackslash}X @{}
    }

    % --- キャプションの設定 ---
    \caption{空中からの水深写真測量(Photogrammetric Bathymetry)における屈折補正手法の比較} 
    \label{tab:refraction_methods_comparison} \\
    
    % --- 1ページ目のヘッダー ---
    \toprule
    \textbf{Category} & \textbf{Reference \& Approach} & \textbf{Workflow Core Mechanism} & \textbf{Performance \& Limitations} \\ 
    \midrule
    \endfirsthead

    % --- 2ページ目以降のヘッダー(続・... と表示させるなど) ---
    \toprule
    \textbf{Category} & \textbf{Reference \& Approach} & \textbf{Workflow Core Mechanism} & \textbf{Performance \& Limitations} \\ 
    \midrule
    \endhead

    % --- ページ最下部のフッター ---
    \midrule
    \multicolumn{4}{r}{\footnotesize (次ページへ続く)} \\
    \endfoot

    % --- 最終ページのフッター ---
    \bottomrule
    \multicolumn{4}{l}{\footnotesize ※ 脚注}
    \endlastfoot

    % ================= コンテンツ =================

    % --- Post-Processing ---
    \textbf{Post-Processing} 
    & 
    \textbf{Woodget et al.} \cite{Woodget2014_PhotograBathy-multiply-n} 
    & 
    \textbf{Simple Refraction Correction} \par
    SfMで得られた見かけの水深に対し、水の屈折率($n \approx 1.34$)を乗数として適用し、一律に深度を補正する。
    & 
    \textbf{課題:}
      \begin{itemize}[leftmargin=1.2em, nosep]
        \item 観測角を考慮せず直下視と仮定するため、画像端部で誤差が増大。
        \item 平坦な水面を仮定。
      \end{itemize} \\ 
    
    \cmidrule{2-4} 
    
    &
    \makecell[l]{\textbf{BathySfM} \\ (Dietrich et al. \cite{Dietrich2016_multi-angle-correction})} 
    & 
    \textbf{Iterative View Dependant} \par
    SfM後の点群に対し、カメラ位置から水面を通してレイを追跡し、スネルの法則に基づいて反復的に真の位置を計算する。
    &
    \textbf{利点:} 視野角による影響を考慮可能。\par
    \textbf{課題:}
      \begin{itemize}[leftmargin=1.2em, nosep]
        \item 水平変位の補正が不十分な場合がある。
        \item 反復計算により処理コスト増。
      \end{itemize} \\ 
    
    \midrule
  
    % --- Image-based ---
    \textbf{Image-based} & 
    \textbf{Agrafiotis et al.} \cite{Agrafiotis2020_refractiveCorrection} 
    & 
    \textbf{画像ベースの屈折補正} \par
    SfM処理を行う\textbf{前}に、生画像に対して屈折除去の幾何変換を行い、屈折フリーな画像を生成してから3D再構成を行う。
    & 
    \textbf{利点:} 既存のSfMパイプラインを利用可能。\par
    \textbf{課題:}
      \begin{itemize}[leftmargin=1.2em, nosep]
        \item 穏やかな水面条件と、海底テクスチャに強く依存。
      \end{itemize} \\ 
    
    \midrule
  
    % --- Machine Learning ---
    \makecell[l]{\textbf{Machine} \\ \textbf{Learning}}
    & 
    \makecell[l]{\textbf{BathyNet} \\ (Mandlburger et al. \cite{Mandlburger2021_BathyNet})} 
    & 
    \textbf{Radiometric Depth Estimation} \par
    マルチスペクトル画像の反射輝度と、ALBの教師データを用いてCNNにより画素単位の水深を推定。
    &
    \textbf{利点:} 非線形な深度推定モデルを獲得可能。\par
    \textbf{課題:}
        \begin{itemize}[leftmargin=1.2em, nosep]
            \item 高品質な教師データ(ALB)が不可欠。
            \item 未学習環境への汎化性能が限定的。
        \end{itemize} \\ 
        

    \midrule
    
    % --- Refraction-Aware ---
    \makecell[l]{\textbf{Refraction-} \\ \textbf{Aware}}
    &   
    \makecell[l]{\textbf{R-SfM} \\ (Makris et al. \cite{Makris2024_refractive-aware-sfm})}     
    & 
    \textbf{Refractive Bundle Adjustment} \par
    スネルの法則をバンドル調整(BA)に組み込み、カメラ外部標定と海底座標を同時最適化。
    & 
    \textbf{利点:} 幾何学的に厳密。カメラと海底の両方を補正。\par
    \textbf{課題:}
      \begin{itemize}[leftmargin=1.2em, nosep]
        \item 密な点群生成にはDeep Learning依存が残る。
      \end{itemize} \\ 

    \cmidrule{2-4} 
    
    &
    \makecell[l]{\textbf{NeRFrac} \\ (Zhan et al. \cite{Zhan2023CVPR_NeRFrac})}     
    & 
    \textbf{NeRF with Refractive Surface} \par
    NeRFを拡張し、屈折を伴う非線形なレイ追跡を行うことで3D形状を再構成。
    &
    \textbf{利点:} 波のある水面(Non-planar)も考慮可能。\par
    \textbf{課題:}
        \begin{itemize}[leftmargin=1.2em, nosep]
            \item 計算コストが極めて高い。
        \end{itemize} \\

\end{xltabular}
\endgroup
