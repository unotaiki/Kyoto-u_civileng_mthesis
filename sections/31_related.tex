%!TEX root = ../main.tex
\section{関連研究}

\note{
  いろいろなPhotogrammetric Bathymetry手法を達観。
  表で表すと見やすいか。。。
}



\url{https://www.notion.so/Photommetric-Bathymetry-2e002f0efa178010b4f3f365fdea4f29}
\cite{Kujawa2025_survey-shallow-bathymetry}からの引用。

\begin{table}[htbp]
  \centering
  \small % Reduces font size slightly to fit more content
  \caption{Comparison of Refraction Correction Approaches for Bathymetric Remote Sensing}
  \label{tab:refraction_methods}
  \renewcommand{\arraystretch}{1.5} % Increases row height for readability
  
  % Define column types:
  % l = left align (auto width)
  % X = auto-wrap paragraph (variable width)
  \begin{tabularx}{\textwidth}{@{} l >{\RaggedRight}X >{\RaggedRight}X @{}}
  \toprule
  \textbf{Reference} & \textbf{Methodological Workflow} & \textbf{Limitations \& Performance} \\ 
  \midrule
  
  % Row 1: Woodget et al.
  \textbf{Woodget et al. [49]} & 
  \textbf{Simple Refraction Correction (Nadir)}
  \begin{itemize}[leftmargin=*, nosep, after=\vspace{\baselineskip}]
      \item Creates a Water Surface Model (WSM) from orthophotos and Digital Elevation Models (DEMs).
      \item Estimates apparent depth: (DEM Base $-$ WSM).
      \item Applies Snell's law with pure water refractive index (1.34).
      \item Corrects DEM by applying the difference between estimated and corrected depths.
  \end{itemize} & 
  \textbf{Limitations:}
  \begin{itemize}[leftmargin=*, nosep]
      \item Assumes a flat water surface.
      \item Reduced effectiveness at depths $>$ 0.4--0.7\,m.
  \end{itemize} 
  \vspace{4pt} \textit{Note: Implementation details in [50].} \\ 
  \midrule
  
  % Row 2: Dietrich
  \textbf{Dietrich [51]} & 
  \textbf{Geometric Refraction (Multi-camera)}
  \begin{itemize}[leftmargin=*, nosep, after=\vspace{\baselineskip}]
      \item Uses SfM processing to find ground coordinates of IFOV and camera corners.
      \item Reconstructs water surface via GNSS edge measurement and point cloud digitization.
      \item Calculates apparent depth and angle of incidence (Snell's Law) per point.
      \item Derives true depth geometrically based on the shift between true and apparent positions.
  \end{itemize} & 
  \textbf{Performance:}
  \begin{itemize}[leftmargin=*, nosep]
      \item Accuracy: $\pm$0.01\,m (0.02\% flight alt).
      \item Precision: 0.06--0.08\,m.
  \end{itemize}
  \vspace{4pt}
  \textbf{Limitations:}
  \begin{itemize}[leftmargin=*, nosep]
      \item Requires flat surface and precise camera orientation.
      \item Best for clear, shallow water (min. turbidity).
  \end{itemize} 
  \textit{Tool: pyBathySfM.} \\ 
  \midrule
  
  % Row 3: Agrafiotis et al.
  \textbf{Agrafiotis et al. [52]} & 
  \textbf{Image-based Refraction Correction}
  \begin{itemize}[leftmargin=*, nosep, after=\vspace{\baselineskip}]
      \item Generates dense point clouds via SfM-MVS from UAS data.
      \item Recovers depth using \textit{DepthLearn} machine learning technique [26,53].
      \item Corrects raw images for refraction using transformation and resampling.
      \item Reprocesses corrected images to generate a refraction-free 3D coastal model.
  \end{itemize} & 
  \textbf{Limitations:}
  \begin{itemize}[leftmargin=*, nosep]
      \item Relies on calm environmental conditions and clear water.
      \item Requires textured seabed (homogeneous areas challenge SfM-MVS).
  \end{itemize} \\ 
  \midrule
  
  % Row 4: Mandlburger et al.
  \textbf{Mandlburger et al. [54]} & 
  \textbf{BathyNet (Deep Neural Network)}
  \begin{itemize}[leftmargin=*, nosep, after=\vspace{\baselineskip}]
      \item Classifies Airborne LiDAR Bathymetry (ALB) points to generate a Digital Water Surface Model (DWSM).
      \item Corrects ALB bottom points for refraction and travel time.
      \item Intersects DWSM rays with DTM to determine oblique water distance.
      \item Trains a U-Net CNN using RGBC multispectral info and calculated distances.
  \end{itemize} & 
  \textbf{Performance:}
  \begin{itemize}[leftmargin=*, nosep]
      \item Bias accuracy: $<$ 15\,cm.
      \item Std. Dev: $\approx$ 40\,cm.
  \end{itemize}
  \vspace{4pt}
  \textbf{Limitations:}
  \begin{itemize}[leftmargin=*, nosep]
      \item Dependent on high-quality reference data (ALB).
      \item Limited generalizability (tested in clear lakes).
  \end{itemize} \\ 
  \bottomrule
  \end{tabularx}
\end{table}
