%!TEX root = ../main.tex
\chapter{APPENDIX}

\section{見かけの深度 (Apparent Depth) の導出}\label{sec:apendix-apparent-depth}

カメラ座標系においてカメラ位置を$z$軸上$\left(0, 0, H\right)$に、また水面を平坦な$xy$平面(法線ベクトルが一定、すなわち静水面とする)に配置する。
水中に存在する各3次元Gaussianプリミティブの位置$\bm{p}$は$\left(x, y, z\right)$(ただし$z<0$)で表される。
簡単化のため、問題を2次元化し、位置を$r = \sqrt{x^2 + y^2}$として$rz$-平面上に射影する(\cref{fig:rz}を参照)。

幾何光学原理およびスネルの法則(Snell's law)より、水中点$\bm{p}$を観測するカメラ視点から見た「見かけの位置」は、観測方向に依存せず定まる\cite{nassar1994_ApparentDepth,Missailidis2025_apparentDepth-leading}。
ここではその詳細な導出を示す。

Gaussian中心の見かけの位置を$\bm{p}'(r', z')$とする。
カメラ位置と見かけの位置$\bm{p}'$を結ぶ入射線と水面との交点を$(s, 0)$($0 < s < r$)とおく。

スネルの法則(空気屈折率$1$、水屈折率$n$、入射角$\theta_i$、屈折角$\theta_r$)は次式で表される。
\begin{equation}
  \sin \theta_i = n \sin \theta_r
\end{equation}

各三角関数項は、カメラ、交点、原点などで構成される三角形$\triangle OAI$, $\triangle IPC$により幾何的に表現できる。

上記関係から、$s$に関する四次方程式が導かれる。
\begin{equation}
\begin{split}
  (1-n^2)s^4 &+ 2(n^2-1)rs^3 \\
  &+ ((1-n^2)r^2 - n^2H^2 + z^2)s^2 \\
  &+ 2n^2H^2rs - n^2H^2r^2 = 0
\end{split}
\end{equation}
(実装ではニュートン法等で数値的に解く。)

わずかなずれ$\Delta s$を仮定し、$s+\Delta s$の交点、および$\theta_r+\Delta \theta_r$の微小変化を考える。三角形$\triangle IP'C'$の関係より、
\begin{equation}\label{eq:triangle-IP'C'}
  -z' \tan \theta_i  = r' - s 
\end{equation}
また、$\triangle I'P'C'$より
\begin{align}\label{eq:triangle-I'P'C'}
  -z' \tan \left(\theta_i + \Delta \theta_i \right) & = (r' - s) + (- \Delta s) \notag \\
  -z' \left( \sin \theta_i + \Delta \theta_i \cos \theta_i  \right) & = (r' -s - \Delta s) \left( \cos \theta_i - \Delta \theta_i \sin \theta_i \right) 
\end{align}

\cref{eq:triangle-IP'C'}と\cref{eq:triangle-I'P'C'}から$z'$を消去し、$\Delta \to 0$の極限をとると
\begin{equation}\label{eq:limit-r'}
  r' = s - \sin \theta_i \cos \theta_i \frac{d s}{d \theta_i}
\end{equation}
が得られる。

また、\cref{eq:triangle-IP'C'}と上式より
\begin{equation}\label{eq:z'_ds/dtheta_i}
  z' = \frac{d s}{d \theta_i} {\cos \theta_i}^2
\end{equation}

さらに、三角形$\triangle IPC$から
\begin{equation}\label{eq:r-s}
  r - s = - z \tan \theta_r
\end{equation}

これを$\theta_r$で微分すると
\begin{equation}\label{eq:ds/dtheta_r}
  \frac{d s}{d \theta_r} = z \cdot \frac{1}{{\cos \theta_r}^2}
\end{equation}

また、スネルの法則\eqref{eq:snell}を$\theta_i$で微分すると
\begin{equation}\label{eq:dtheta_r/dtheta_i}
  \frac{d \theta_r}{d \theta_i} = \frac{1}{n} \cdot \frac{\cos \theta_i}{\cos \theta_r}
\end{equation}

\eqref{eq:dtheta_r/dtheta_i},\eqref{eq:ds/dtheta_r}より
\begin{equation}\label{eq:ds/dtheta_i}
  \frac{d s}{d \theta_i} = \frac{z}{n} \cdot \frac{\cos \theta_i}{{\cos \theta_r}^3}
\end{equation}

以上をまとめると、見かけの位置$(r', z')$はそれぞれ
\begin{equation}
  \begin{split}
    (1-n^2)s^4 &+ 2(n^2-1)rs^3 \\
    &+ ((1-n^2)r^2 - n^2H^2 + z^2)s^2 \\
    &+ 2n^2H^2rs - n^2H^2r^2 = 0
  \end{split}
\end{equation}
と表される。
