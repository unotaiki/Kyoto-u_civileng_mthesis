%!TEX root = ../main.tex

\subsection{Opacity Correction}\label{sec:opacity-correction}

ここで、述べる不透明度補正(Opacity Correction)は、\cref{sec:color-correction}で述べた放射輝度補正の前手法であり、放射輝度補正ほど正確な物理モデルに基づいた補正ではない。
しかし、実装の検証において、用いたためここで記す。

屈折界面を通して観測されるシーンの再構成では、特有の放射輝度減衰が生じる。
\cref{fig:refraction-makes-scene-darker}に示すように、水面越しに見えるシーンはエタンドゥ保存則 (etendue conservation) により著しく暗く見える \cite{etendue}。
屈折率 $n$ の媒質から空気へ光束が移動する場合、放射輝度は次式でスケーリングされる。
\begin{equation}
L_{\mathrm{air}} = \frac{1}{n^{2}} L_{\mathrm{water}}
\end{equation}
この観測輝度の低下は 3D Gaussian Splatting にとって課題となる。
暗い観測画像に直接最適化すると、解が不自然に低い不透明度へと崩壊しやすい。
半透明なガウシアンは低い観測輝度を説明できる一方で、幾何が曖昧でぼやけた、拘束の弱い再構成を引き起こす。

これらを抑制するため、上記の放射輝度関係に着想を得た不透明度の正則化を導入する。
最適化が不透明度を自由に下げることを許すのではなく、屈折界面の背後に位置するガウシアンの不透明度を次式でスケーリングし、期待される輝度損失を補正する。
\begin{equation}
\alpha' = \frac{1}{n^{2}} \alpha
\end{equation}
この調整により各ガウシアンの不透明度の上限が抑えられ、低不透明度への自明な収束が抑制される。
結果として、より密で空間的に整合的なプリミティブでシーンを表現するように最適化が誘導され、幾何の忠実度と見かけの忠実度が向上する。

なお、本戦略は屈折の物理的に厳密なモデルではない。
屈折環境における最適化を安定化するためのヒューリスティクスである。
実験的には、よりシャープな幾何が一貫して得られ、フローターや過度な平滑化といった典型的なアーティファクトを回避できる。
