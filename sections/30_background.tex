%!TEX root = ../main.tex
\chapter{研究背景と関連研究}\label{chap:background}
\note{浅水域を具体的に定義する文章を挿入。水底が視認可能な水域で、現実的には透明度が高く水面が穏やかな水域。
具体的には、淡水では~2m、海水では~5m程度が現実的。
}

% ========================
% ==== 浅水域水深測量の重要性 ===== 
% ========================

地球表面のおよそ7割は、海や河川といった水域によって占められる。
近年、日本財団\cite{NipponFoundation}とGEBCO(General Bathymetric Chart of the Oceans)\cite{GEBCO}による国際プロジェクト「Seabed 2030」\cite{Seabed2030}の推進によって、深海域における海底地形図の整備は急速に進展している\cite{Trethewey2025_book-bathymetry}。
対照的に、陸域と水域の境界である「浅水域(Shallow Water)」は、既存の測量技術では効率的な計測が難しく、水路測量分野において`White Ribbon'と呼ばれるデータの空白地帯となっている。
これは、海図や河川図において、陸域の地形データと深部の水深データとの間に挟まれ、有効な測量データが存在しないために白く描かれ、帯状をなすことに由来する。 

一方、浅水域は、人間社会の経済活動、防災、そして生態系保全において決定的な役割を果たす領域である。
本章では、浅水域における三次元計測の重要性を、河川管理および生態学(エコトーン)の観点から概説し、現状の課題を整理する。


\begin{figure}[htbp]
  \centering
  \includegraphics[width=0.6\textwidth]{figure/30_/white-ribbon.png}
  \caption{
    White Ribbonの実例。
    \cite{Mcmahon2012_white-ribbon-figure}より引用。
    \cite{GEBCO}の提供するGridded Bathymetry Dataによる海洋沿岸の水深測量図。
    白い領域は浅水域のため測量不能。
    \note{White Ribbonを強調して示せるように、他の図を半透明に、White Ribbonを囲み強調する}
    }
    \label{fig:white-ribbon-figure}
\end{figure}

\section{浅水域における水深測量の重要性}\label{sec:bg_application}

\subsubsection{定期横断測量}\label{subsec:bg_appli_japan-river-survey}
河川および沿岸域の浅瀬は、洪水の流下能力を決定する重要な断面であり、津波や高潮に対する第一の防衛線として機能する。
特に日本においては、河川法に基づき、一級河川を対象とした定期的な横断測量が義務付けられている\cite{river-survey-milt}。

しかし、現行の定期横断測量には、空間分解能と安全性において重大な課題が存在する。
標準的な測量間隔は200メートルピッチとされており、断面間の局所的な地形情報は欠落する。
また、従来の水部計測は、有人船による音響測深や、測量員が直接川に入りポールで計測する手法に依存している。
これらは、急流河川における事故のリスクの危険を伴うだけでなく、測線維持のための慎重な操船や作業を要するため、広範囲のデータ取得において著しく時間的効率を欠いている。
したがって、河川マネジメント上重要であるにも関わらず、浅水域の詳細な地形情報は依然として空間解像度を欠く。

\begin{figure}[htbp]
  \centering
  \includegraphics[width=0.6\textwidth]{figure/30_/manual_river_survey.png}
  \caption{
    一級河川縦横断測量の実例。
    \cite{fig:manual_river_survey}より引用。
    測量員によるポールによる計測には多大な労力と時間がかかる一方、疎な三次元情報しか得られない。
    }
    \label{fig:manual-river-survey}
\end{figure}

\subsubsection{河川地形の時系列変化モニタリング}\label{subsec:bg_appli_seigyu-morphological-change}
河川工学の観点からは、浅水域計測には「高解像度」に加え、「高頻度」なモニタリングが求められる。
急峻な地形を有し、台風等による豪雨が頻発する日本の河川では、土砂移動が活発であり、出水のたびに局所地形が大きく変化する。
そのため、5年に1度といった従来の定期測量サイクルでは、洗掘や堆積といった動的な地形変化を捉えることができない。

この課題は、伝統的河川工法である「聖牛」などの機能評価において顕著に現れる。
聖牛は丸太を組み上げた構造物であり(\cref{fig:seigyu})、河川砂州部に設置することで流速制御や砂州などの生息地形成を図る。
「Nature-Based Solutions (NbS)」として近年再評価されている技術である。
\cite{Fujii2024_seigyu}は、京都府木津川における6年間にわたるモニタリングを通じ、聖牛設置地域一帯を観測し、聖牛の設置によって地形が大きく変化することを明らかにした(\cref{fig:seigyu-morphological-change})。
同研究が示すように、砂州が単調な地形から水路や池が点在する複雑な地形へと遷移する過程を定量的に評価するためには、陸域と水域をシームレスに、かつ高解像度で捉える三次元計測技術が必要不可欠である。

\begin{figure}[htbp]
  \centering
  \includegraphics[width=0.6\textwidth]{figure/30_/Seigyu.jpg}
  \caption{
    聖牛の写真。
    \cite{Fujii2024_seigyu}より引用。
    }
    \label{fig:seigyu}
\end{figure}

\begin{figure}[htbp]
  \centering
  \includegraphics[width=0.95\textwidth]{figure/30_/2024Fujii_morphological-change-by-seigyu.jpg}
  \caption{\cite{Fujii2024_seigyu}より引用。聖牛による河川地形の時系列変化。}
  \label{fig:seigyu-morphological-change}
\end{figure}



\subsubsection{エコトーン}\label{subsec:bg_appli_ecotone}
浅水域の重要性は、防災のみならず生態系保全の観点からも極めて高い。
20世紀初頭に提唱された「エコトーン(Ecotone)」は、二つの異なる生態系が接し移行する境界領域を指し\cite{Ecotone}、浅水域においては陸域から水域へと移行する水際線がこれに相当する。
この領域は、陸と水双方の環境要因が作用することで、高い生物多様性を有する「エッジ効果(Edge Effect)」が発現する場として知られる\cite{}。

エコトーンの形状と生態学的プロセスの関係については、多くの研究がなされている。
例えば、\cite{Casalini2019_geo-ecological_patagonia_land}は乾燥地帯の植生分布が旧河道などの微地形に依存することを示し、\cite{Perry2018_geo-ecological_coral-reefs}は珊瑚礁の三次元構造が生物侵食速度や多様性に与える影響を報告している。
\checkref{陸域は説得力がない。もっとストレートな研究を探す。エコトーンというワードだと出てこないかも。Geomorphologyとかで調べる}
これらの研究は、エコトーンの環境保全において、詳細な三次元地形情報が不可欠であることを示唆している。

% ここで、本研究が提案する写真測量(Photogrammetry)との重要な関連性を指摘する。
浅水域エコトーンにおける生態系の豊かさは、太陽光の到達に由来する。
水生植物や藻類の基礎生産が行われる水深帯は「有光層(Photic Zone)」と呼ばれ、水底まで十分な光が届くことが生物生息の条件となる。
一方、写真測量は受動的光学手法(Passive Optical Method)であり、原理的に水底からの反射光を必要とするため、濁度や水深による制約を受ける手法である。
従って、「生物多様性が高く計測ニーズのある浅水域」は、必然的に「光が水底まで達する写真測量が可能な場所」であると言える。
深すぎて光が届かない領域は、従来の音響測深が有効な領域であるが、浅水域エコトーンにおいては、光学的計測は合理的なアプローチとなる。

また、エコトーンの理解には水域だけでなく連続する陸域の植生と地形の把握も同時に求められるため、水中のみを計測するUSV等による音響測深は適さない。
\note{植生 = フォトリアルな見た目 から判別というニュアンスを含め GSのNVSタスクの有用性を後に説明したい}
以上の点から、陸域と浅水域を同時に、かつ高密度に計測可能な空中写真測量は、エコトーンの地形計測(Geomorphology)において最適な手法であると位置付けられる。


\IncludeTwoImages[4cm]
  {figure/30_/ecotone.jpg}{}
  {figure/30_/swamp.jpg}{}
  {エコトーンの例: :浅水域一帯では、陸域と水域の地形中に多様な植生が見られる。}
  {fig:ecotone-eg}

\begin{figure}[htbp]
  \centering
  \includegraphics[width=0.70\textwidth]{figure/30_/Ecotone_illustration.png}
  \caption{
    エコトーンの模式図。\cite{fig:Ecotone}より引用。
    陸域と水域の境界にあたる浅水域は、双方の生態系が重なり合う重要な領域である。
    }
  \label{fig:ecotone-illustration}
\end{figure}
  




% ========================
% ==== 水深測量の諸手法 ===== 
% ========================
\section{既存の水深測量手法とその課題}\label{sec:bg_bathymetry-method-overview}
本節では、既存の水深測量技術を概観し、浅水域(Shallow Water)の三次元計測において生じる技術的課題を整理する。

\subsection{音響測深 (Echo sounding)}\label{subsec:bg_sonar}


現在、水深測量(Bathymetry)の標準となっているのは、船舶に搭載したマルチビーム音響測深機(Multibeam Echosounder: MBES)である。
MBESは船底から扇状(Fan-shape)に音波を発射し、走査線上の多数の計測点の水深を同時に取得することで、面的な地形図を作成する。
深海域においては、一度の航行で数キロメートル幅の海底をスキャンできるため、Seabed 2030のような全地球的海洋マッピングプロジェクトの中核技術となっている。

しかし、浅水域においてMBESの計測効率は劇的に低下する。
MBESの走査幅(Swath Width: $SW$)は、幾何学的に水深 $D$ と指向角 $\theta$(一般に$120^{\circ} \sim 150^{\circ}$)に依存して決定されるためである。
\cref{eq:swath-width}が示すように、走査幅は水深に比例する:
\begin{equation}\label{eq:swath-width}
  SW = 2D \tan \left( \frac{\theta}{2} \right)
\end{equation}
水深 \qty{1000}{m} であれば \qty{3000}{m} 以上の幅を一度に計測可能であるが、日本の河川のような水深 \qty{1}{m} $\sim$ \qty{2}{m} の環境では、走査幅はわずか \qty{3}{m} $\sim$ \qty{8}{m} 程度に留まる。
したがって、対象水域を網羅するためには、探査船は数十回もの往復(測線)を繰り返す必要があり、時間的・金銭的コストの増大を招く。
また、送受波器の直下数十センチメートルは、音波の振動の余韻(リンギング)といった制約により、計測が不可能となる
(Teledyne Marine社製: SeaBat T20-Pでは\cref{SeaBat_T20-P_product_leaflet} 深度50センチメートル以浅は、計測が不可能である)。
% トランスデューサーが「ドン!」と音波を発射した後、その振動(余韻)が完全に止まるまでにはわずかな時間がかかります(これをリンギングと呼びます)。振動している間にすぐ近くから反射波が帰ってきても、余韻にかき消されて受信できません。

加えて、物理的な制約も存在する。
有人調査船は、座礁のリスクがあるため船体の喫水より浅い場所には侵入できない。
この課題に対し、近年では小型の無人水上艇(Unmanned Surface Vehicle: USV)を用いた測量も研究されている\cite{Giordano2015_sonar-bathymetry,Kurowski2019_survey-USV}。
USVは浅水域へのアクセスを可能にするが、流水環境下での姿勢制御や自己位置推定の難易度が高く、また走査幅の物理的制約(\cref{eq:swath-width})は浅水域ではより一層深刻になるため、広範囲を高頻度で計測するには依然としてコストが高い。
さらに、音響測深は水域のみに限定されるため、陸域から水域へ連続する地形を単独で計測することは不可能であり、UAV写真測量等とのデータ統合が別途必要となる。

\IncludeThreeImages[3cm]
  {figure/30_/sonar_1.jpg}{マルチビームソナーの概念図}
  {figure/30_/sonar_2.jpg}{マルチビームソナーによる水深測量図}
  {figure/30_/sonar_usv.png}{無人水上艇(USV)による水深測量例}
  {
  マルチビームソナーによる水深測量。
  \cite{Giordano2015_sonar-bathymetry}より引用。
  }
  {fig:sonar-figure}

\subsection{航空レーザ測深 (Airborne LiDAR Bathymetry)}\label{subsec:bg_alb}
航空レーザ測深(Airborne LiDAR Bathymetry: ALB)は、航空機から水を透過する緑色レーザ(波長\qty{532}{nm}付近)を照射し、水面反射と水底反射の時間差から水深を求める技術である\cite{Saylam2018_ALB}。
自らエネルギーを照射して観測を行うため、能動的リモートセンシング(Active Remote Sensing)に分類される。
日本国内においても、国土交通省により一級河川の定期横断測量への導入が進められており\cite{milt_river_3d_application_manual}、陸域と水域をシームレスに、かつ高密度に計測できる点が大きな利点である。
\note{NotionにALBの資料をまとめたことあり}
\note{何mくらいまで測れるか。}

しかし、ALBには本研究が対象とする「浅水域の高頻度モニタリング」において、以下の決定的な欠点が存在する。
\begin{itemize}
  \item \textbf{導入・運用コスト}: 有人航空機や大型ドローンに搭載する高出力LiDARは極めて高価(数千万円規模)であり、運用コストも高い。国家規模で行う5年に1度の定期測量には適しているが、出水のたびに地形変化を追跡するような機動的な運用は経済的に難しい。
  % \item \textbf{不感帯(Dead Zone)の存在}: ALBはパルス波形の解析によって水深を特定するが、水深が極めて浅い場合(一般に\qty{0.3}{m}〜\qty{0.5}{m}以下)、水面反射波と水底反射波が重なり合い、分離が困難となる「不感帯」が生じる。これは、水際線付近の詳細な形状把握において致命的である。
  \item \textbf{水質の影響}: 光学的計測である以上、濁度が高い水域ではレーザが散乱・減衰し、水底まで到達しない範囲では適用不可である。\cite{milt_river_3d_application_manual}によると、2 $\sim$ 3 m程度まで計測可能である。
\end{itemize}

\missingfigure{Passive Remote SensingとActive Remote Sensingの図。}
\missingfigure{各手法のPros \& Cons をTableにする。手法、測定可能深度、空間解像度、浅水域での効率、コスト、頻度、安全性、備考}



\subsection{Spectrally Derived Bathymetry}\label{subsec:bg_sdb} 
分光水深測量(Spectrally Derived Bathymetry: SDB)は、LandsatやSentinel-2などのマルチスペクトル衛星画像の分光特性を利用し、放射伝達モデルに基づいて水深を推定する手法である\cite{He2024_survey-shallow-bathymetry}。
\checkref{SDBに関して特化したSurvey論文があったように思える}
本手法は受動的リモートセンシング(Passive Remote Sensing)に分類され、広域かつ低コストに水深情報を取得できる利点がある一方で、その推定原理に起因する理論的・実用的な制約が存在する。

SDBの基本原理は、水中における光の指数関数的な減衰に基づく。
可視光領域において、水深が深いほど水底からの反射光は水柱による吸収・散乱を受けて減衰する。
SDBはこの物理現象を利用し、主に青~緑バンド(水中透過率が高い)と赤~近赤外バンド(水中減衰が大きい)の輝度比や、各バンドの反射率の対数線形モデル(Lyzenga法やStumpf法など)を用いて水深を回帰的に推定する。
\checkref{Lyzenga法やStumpf法などの引用}

SDBは海洋沿岸部等の広域な浅海域では有効な手法である一方、中小河川環境への適用においては、幾何学的アプローチである写真測量と比較して以下の本質的な限界を有する。

\begin{itemize} 
  \item \textbf{放射量依存性と水質・底質の不均一性}: SDBは幾何学的な三次元復元ではなく、観測された放射輝度(Radiance)に基づく放射量的な推定(Radiometric estimation)である。
  そのため、推定精度は水域に固有の光学的特性(濁度やクロロフィルa濃度)や、水底の底質(砂、礫、植生)によるアルベドの空間的不均一性に強く依存する。
  
  \item \textbf{経験的モデルと現地データの必要性}: 一般的なバンド比法などの経験的モデルでは、輝度値を水深値へ変換するために現地での実測水深データを用いたキャリブレーション(回帰モデルの補正)が不可欠である。
  これは「計測なしで水深を得る」というリモートセンシングの利点を部分的に損なうものである。
  
  \item \textbf{空間解像度とミクセル問題}: Sentinel-2(\qty{10}{m})やLandsat(\qty{30}{m})などのオープンデータである衛星画像は、中小河川における三次元情報のニーズに対して解像度が不足する。
  1ピクセル内に水域、陸域、河畔林が混在する「ミクセル(Mixed Pixel)」問題が発生し、水深推定を阻害する。 
\end{itemize}


\begin{figure}[htbp]
  \centering
  \includegraphics[width=0.95\textwidth]{figure/30_/Kujawa2025_RS-bathymetry-methods.png}
  \caption{
    リモートセンシングによる水深測量技術の分類。
    \cite{Kujawa2025_survey-shallow-bathymetry}より引用。
    (a) 能動的・受動的手法の分類。(b) 各手法の空間分解能とカバレッジの関係。
  }
  \label{fig:remote-sensing-bathymetry-method}
\end{figure}


\subsection{空中写真測量 (Photogrammetric Bathymetry)}\label{subsec:bg_photogrammetric-bathymetry-overview}
以上の既存手法の課題を踏まえ、近年急速に普及したドローンなどの無人航空機(Unmanned Aerial Vehicle: UAV)用いた写真測量(Photogrammetry)は、低コスト・高解像度・高頻度なデータ取得が可能であることから、浅瀬測量技術においても有効であると考えられる。
UAVにより空撮された多視点画像から、Structure-from-Motion (SfM) \cite{schoenberger2016_colmap} および Multi-View Stereo (MVS) \cite{Furukawa2010_PatchMVS,Furukawa2015_MVS} を用いて3次元形状を復元するアプローチは、陸部においては既に広い用途で実用化されている\cite{Bemis2014_UAV-photogrammetry,Gomez2016_UAV-photogrammetry-disaster,Iglhaut2019_UAV-photogrammetry-forestry}。
この技術を水中に適用する場合、その手法は「空中からの水深写真測量(Photogrammetric Bathymetry)」と呼ばれ、多視点画像からの水中三次元再構成問題として定式化される。
しかし、これを実現するためには、光の反射や、水中での光の散乱・吸収による減衰、波による被写体の歪みなどの課題が存在するが、中でも最も根本的で重要な課題として、水面における「光の屈折(Refraction)」を克服しなければならない。
\missingfigure{図で屈折が既存の光の直進性やPinholeモデルを成立させなくことを示す}

既存のSfM/MVSアルゴリズムの大部分は、カメラと被写体が同一媒質(空気中)にあることを前提とし、光が直進するという幾何的な仮定に基づいている。
しかし、上空からの水中撮影においては、光は水中から空気中へ進む際に、媒質境界(水面)でスネルの法則(Snell's Law)に従って屈折する。
この屈折現象により、カメラから観測される被写体の「見かけの位置(Apparent Position)」は、実際の位置よりも浅く、かつ歪んで観測される。
したがって、従来の陸上用アルゴリズムをそのまま適用すると、水深が過小評価され、復元形状が破綻する。
本研究では、この屈折を物理的正確に考慮した(Refraction-aware)新たな三次元再構成手法を導入することで、水面下の高精度な計測を実現する。
