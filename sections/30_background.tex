%!TEX root = ../main.tex
\chapter{研究背景と関連研究}\label{chap:background}
\note{浅水域を具体的に定義する文章を挿入。水底が視認可能な水域で、現実的には透明度が高く水面が穏やかな水域。
具体的には、淡水では~2m、海水では~5m程度が現実的。
}

% ========================
% ==== 浅水域水深測量の重要性 ===== 
% ========================
\section{研究背景:浅水域における水深測量の重要性}\label{sec:bg_application}

地球表面のおよそ7割は、海や河川といった水域によって占められる。
近年、日本財団\cite{NipponFoundation}とGEBCO(General Bathymetric Chart of the Oceans)\cite{GEBCO}による国際プロジェクト「Seabed 2030」\cite{Seabed2030}の推進によって、深海域における海底地形図の整備は急速に進展している\cite{Trethewey2025_book-bathymetry}。
対照的に、陸域と水域の境界である「浅水域(Shallow Water)」は、既存の測量技術では効率的な計測が難しく、水路測量分野において`White Ribbon'と呼ばれるデータの空白地帯となっている\cite{Mason2008_colouring-the-white-ribbon}。
これは、海図や河川図において、陸域の地形データと深部の水深データとの間に挟まれ、有効な測量データが存在しないために白く描かれ、帯状をなすことに由来する。 

一方、浅水域は、人間社会の経済活動、防災、そして生態系保全において決定的な役割を果たす領域である。
本章では、浅水域における三次元計測の重要性を、河川管理および生態学(エコトーン)の観点から概説し、現状の課題を整理する。


\begin{figure}[htbp]
  \centering
  \includegraphics[width=0.6\textwidth]{figure/30_/white-ribbon_attention.png}
  \caption{
    White Ribbonの実例。
    \cite{Mcmahon2012_white-ribbon-figure}より引用。
    \cite{GEBCO}の提供するGridded Bathymetry Dataによる海洋沿岸の水深測量図。
    赤枠で囲われた白い領域は浅水域のため測量不能。
    }
    \label{fig:white-ribbon-figure}
\end{figure}


\subsubsection{定期横断測量}\label{subsec:bg_appli_japan-river-survey}
河川および沿岸域の浅瀬は、洪水の流下能力を決定する重要な断面であり、津波や高潮に対する第一の防衛線として機能する。
特に日本においては、河川法に基づき、一級河川を対象とした定期的な横断測量が義務付けられている\cite{river-survey-milt}。

しかし、現行の定期横断測量には、空間分解能と安全性において問題が存在する。
標準的な測量間隔は200メートルピッチとされており、断面間の局所的な地形情報を捉えることができない。
また、従来の水部計測は、有人船による音響測深や、測量員が直接川に入りポールで計測する手法に依存している(\cref{fig:manual-river-survey})。
これらは、水難事故のリスクを伴うだけでなく、測線維持のための慎重な操船や作業を要するため、広範囲のデータ取得において著しく時間的効率を欠いている。
こうした課題は、少子高齢化による作業員の不足や、河川管理に面的な管理が求められはじめている状況も鑑みて、
効率的で高密度な水深測量技術の開発が重要になる。


\begin{figure}[htbp]
  \centering
  \includegraphics[width=0.6\textwidth]{figure/30_/manual_river_survey.png}
  \caption{
    一級河川縦横断測量の実例。
    \cite{fig:manual_river_survey}より引用。
    測量員によるポールによる計測には多大な労力と時間がかかる一方、疎な三次元情報しか得られない。
    }
    \label{fig:manual-river-survey}
\end{figure}

\subsubsection{河川地形の時系列変化モニタリング}\label{subsec:bg_appli_seigyu-morphological-change}
河川工学の観点からは、浅水域計測には「高解像度」に加え、「高頻度」なモニタリングが求められる。
急峻な地形を有し、台風等による豪雨が頻発する日本の河川では、土砂移動が活発であり、出水のたびに局所地形が大きく変化する。
そのため、5年に1度といった従来の定期測量サイクルでは、洗掘や堆積といった動的な地形変化を捉えることができない。

この課題は、伝統的河川工法である「聖牛」などの機能評価において顕著に現れる。
聖牛は丸太を組み上げた構造物であり(\cref{fig:seigyu})、河川砂州部に設置することで流速・流域を制御し、人間社会に寄与する災害の起こりにくい地形形成を導く、「Nature-Based Solutions (NbS)」として、近年再評価されている技術である。
砂州などの自然地形が形成されることで、これらは生物の生息地となり、生態系の多様性の基盤ともなる。
\cite{Fujii2024_seigyu}は、京都府木津川における6年間にわたるモニタリングを通じ、聖牛設置地域一帯を観測し、聖牛の設置によって地形が大きく変化することを明らかにした(\cref{fig:seigyu-morphological-change})。
同研究が示すように、砂州が単調な地形から水路や池が点在する複雑な地形へと遷移する過程を定量的に評価するためには、陸域と水域をシームレスに、かつ高解像度で捉える三次元計測技術が必要不可欠である。

\begin{figure}[htbp]
  \centering
  \includegraphics[width=0.6\textwidth]{figure/30_/Seigyu.jpg}
  \caption{
    聖牛の写真。
    \cite{Fujii2024_seigyu}より引用。
    }
    \label{fig:seigyu}
\end{figure}

\begin{figure}[htbp]
  \centering
  \includegraphics[width=0.95\textwidth]{figure/30_/2024Fujii_morphological-change-by-seigyu.jpg}
  \caption{\cite{Fujii2024_seigyu}より引用。聖牛による河川地形の時系列変化。}
  \label{fig:seigyu-morphological-change}
\end{figure}



\subsubsection{エコトーン}\label{subsec:bg_appli_ecotone}
浅水域の重要性は、防災のみならず生態系保全の観点からも極めて高い。
20世紀初頭に提唱された「エコトーン(Ecotone)」は、二つの異なる生態系が接し移行する境界領域を指し\cite{Ecotone}、陸域から水域へと移行する浅水域一帯はこれに該当する。
この領域は、陸と水双方の環境要因が作用することで、高い生物多様性を有するエッジ効果(Edge Effect)が発現する場として知られる\cite{}。

エコトーンの形状と生態学的プロセスの関係については、多くの研究がなされている。
例えば、\cite{Casalini2019_geo-ecological_patagonia_land}は乾燥地帯の植生分布が旧河道などの微地形に依存することを示し、\cite{Perry2018_geo-ecological_coral-reefs}は珊瑚礁の三次元構造が生物侵食速度や多様性に与える影響を報告している。
\checkref{陸域は説得力がない。もっとストレートな研究を探す。エコトーンというワードだと出てこないかも。Geomorphologyとかで調べる}
これらの研究は、エコトーンの環境保全において、詳細な三次元地形情報が不可欠であることを示唆している。

浅水域エコトーンにおける生態系の豊かさは、太陽光の到達に由来する。
水生植物や藻類の基礎生産が行われる水深帯は「有光層(Photic Zone)」と呼ばれ、水底まで十分な光が届くことが生物生息の条件となる。
一方、写真測量は受動的光学手法(Passive Optical Method)であり、原理的に水底からの反射光を必要とするため、濁度や水深による制約を受ける手法である。
一方、受動的リモートセンシング(Passive Remote Sensing)である写真測量では、濁度や水深による制約が少なく、水底が確認可能である必要がある。
従って、「生物多様性が高く計測ニーズのある浅水域」は、必然的に「光が水底まで達する写真測量が可能な場所」である可能性が高いと言える。
深すぎて光が届かない領域は、従来の音響測深が有効な領域であるが、浅水域エコトーンにおいては、光学的計測は合理的なアプローチとなる。
また、エコトーンの理解には水域だけでなく連続する陸域の植生と地形の把握も同時に求められるため、水中のみを計測するUSVなどによる音響測深は適さない。
\note{植生 = フォトリアルな見た目 から判別というニュアンスを含め GSのNVSタスクの有用性を後に説明したい}
以上の点から、陸域と浅水域を同時に、かつ高密度に計測可能な空中写真測量は、エコトーンの地形計測(Geomorphology)において最適な手法であると位置付けられる。


\IncludeTwoImages[4cm]
  {figure/30_/ecotone.jpg}{}
  {figure/30_/swamp.jpg}{}
  {エコトーンの例: :浅水域一帯では、陸域と水域の地形中に多様な植生が見られる。}
  {fig:ecotone-eg}

\begin{figure}[htbp]
  \centering
  \includegraphics[width=0.70\textwidth]{figure/30_/Ecotone_illustration.png}
  \caption{
    エコトーンの模式図。\cite{fig:Ecotone}より引用。
    陸域と水域の境界にあたる浅水域は、双方の生態系が重なり合う重要な領域である。
    }
  \label{fig:ecotone-illustration}
\end{figure}
  
