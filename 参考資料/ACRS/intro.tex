\section{Introduction}\label{INTRODUCTION}
\sloppy

\begin{figure*}[htbp]
  \centering
  \captionsetup{justification=raggedright}
  \includegraphics[width=0.9\linewidth]{figure/Intro/overview_task2.pdf}
  \caption{Overview of our proposed method for photogrammetric bathymetry. 
  Our Refractive-Aware Gaussian Splatting takes multi-view images distorted by refraction (top right) as input. 
  By explicitly modeling refraction within the rendering pipeline, it reconstructs a geometrically accurate and refraction-free 3D scene (bottom left).}
  \label{fig:photogrammetric-bathymetry}
\end{figure*}

Bathymetry is critically important.
Accurate topographic information serves as fundamental data for river management, including flood prediction and monitoring geomorphological changes, and it also has potential applications in marine environments as well.
However, sonar-based hydrographic surveying, which is widely used from vessels, faces significant limitations in shallow waters due to the risk of grounding and equipment damage, and narrow and inefficient survey width.  
More recently, Airborne Light Detection and Ranging Bathymetry (ALB) has been employed, but both the equipment and data acquisition costs remain prohibitively high.  
On land, UAV-based photogrammetry has become a widely adopted technique, offering an efficient, inexpensive, safe, and accessible solution.  
Nonetheless, when applied to underwater targets from aerial platforms, photogrammetry encounters fundamental challenges.  
These include specular reflections from the water surface, distortions by waves, and light scattering and attenuation within the water medium.  
Among these factors, optical refraction is particularly detrimental, as it fundamentally breaks the assumption that light travels in a straight line, which is the foundation of photogrammetry.  

A variety of methods have been proposed to mitigate refraction effects, but most rely on post-hoc or iterative corrections applied to the outputs of conventional photogrammetric pipelines.  
Since they lack explicit physical modeling of refraction, such approaches fail to generalize to scenarios involving oblique or close-range imaging, leaving the core problem unresolved.  
Recently, refraction-aware structure-from-motion (R-SfM) methods have emerged \parencite{Makris2024_refractive-aware-sfm}, explicitly incorporating refraction into the SfM pipeline.  
Although promising, these methods typically yield only sparse point clouds and rely on deep learning–based densification techniques trained on ALB-derived datasets \parencite{Agrafiotis2019ISPRS_SVM-UAV-Bathymetry}.  
Given the scarcity of high-quality training data, applying these approaches across diverse regions remains infeasible.  

Concurrently, 3D Gaussian Splatting (3DGS) has recently gained significant attention as an emerging paradigm for 3D reconstruction \parencite{Kerbl2023ToG_3DGS}.  
Building on this advancement, we propose a novel foundation for photogrammetric bathymetry: Refractive-Aware Gaussian Splatting, which directly integrates an optically accurate two-media refraction model into the Gaussian Splatting pipeline.  
Our methods not only enables the recovery of geometrically accurate bathymetry but also provides photorealistic renderings for a clear appearance with refraction artifacts effectively removed.  
The key idea is to incorporate a differentiable parameter transformation that adjusts each Gaussian’s parameters with respect to the camera ray according to appearance depth, embedding this directly into the optimization process of Gaussian Splatting.  

Under the assumption of a calm, planar water surface without reflection and attenuation underwater, 
Our method reconstructs accurate 3D Gaussians free of refraction effect from multi-view images with known camera poses and refractive surface position, initialized with only a coarse prior of the depth of the floor.  
On multiple synthetic CG datasets, our method achieved image quality metrics exceeding PSNR 25 and demonstrated shape reconstruction with F1-scores above 90\% after 3DGS to point cloud conversion.  
This establishes a foundation for a next generation of photogrammetric bathymetry.  

\newpage

