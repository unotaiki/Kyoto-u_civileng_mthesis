The 3D reconstruction of geometry in shallow water has a wide range of applications, such as monitoring bed morphology changes and conducting hazard simulations. 
In addition, the detailed texture of the reconstructed data is useful for bed surface classification and aquatic habitat quantification. 
In this context, bathymetric surveying using photogrammetry from aerial images captured by Unmanned Aerial Vehicles (UAVs) is gaining attention as a particularly efficient method. 
However, it is fundamentally challenged by light refraction at the air-water interface, which invalidates the geometric principles of photogrammetry. 
Existing methods either rely on iterative post-processing or employ deep learning models that lack physical guarantees and explainability. 
We solve this challenge by introducing a refraction-aware 3D Gaussian Splatting framework that incorporates an optically accurate model of two-media refraction directly into the reconstruction pipeline. 
Our key innovation is a differentiable coordinate transformation that analytically models light refraction, mapping 3D Gaussians from their true underwater positions to their apparent space for each aerial view. 
This enables unified optimization, simultaneously solving for dense scene geometry and detailed appearance while maintaining the efficiency of standard 3D Gaussian Splatting. 
We evaluated our method on a simulated UAV dataset of a riverbed, rendered with physically-based ray tracing to isolate refractive effects from other optical phenomena. 
Our approach achieved a geometric F1-score of 96\% (with a 10 cm error threshold at a depth scale of 10 m). 
Furthermore, in novel view synthesis, we obtained photorealistic views with a Peak Signal-to-Noise Ratio (PSNR) of 25.9 dB and a Structural Similarity Index Measure (SSIM) of 0.93. 
By creating 3D models that are both photorealistic in appearance and dense and geometrically precise in structure, our method addresses a key challenge in aquatic remote sensing from aerial imagery. 
This work enables cost-effective, high-frequency monitoring of riverbeds, lakeshores, and seashores under calm surface conditions.