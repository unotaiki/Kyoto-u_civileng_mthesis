\documentclass{isprs}

% プリアンブル (文章の型版を定義)
\usepackage{subfigure}
\usepackage{setspace}
\usepackage{geometry} % added 27-02-2014 Markus Englich
\usepackage{epstopdf}
\usepackage[labelsep=period]{caption}  % added 14-04-2016 Markus Englich - Recommendation by Sebastian Brocks
\usepackage[british]{babel} 
\usepackage[hang]{footmisc}
\def\footnotemargin{1em} % added 08-01-2020 Dennis Wittich

%\usepackage[authoryear]{natbib}
%\def\bibhang{0pt}

\geometry{a4paper, top=25mm, left=20mm, right=20mm, bottom=25mm, headsep=10mm, footskip=12mm} % added 27-02-2014 Markus Englich
%\usepackage{enumitem}

%\usepackage[perpage,para,symbol*]{footmisc}

%\renewcommand*{\thefootnote}{\fnsymbol{footnote}}
\captionsetup{justification=centering,font=normal} % thanks to Niclas Borlin 05-05-2016
\captionsetup[figure]{font=small} % added 23-04-2019 Dennis Wittich
\captionsetup[table]{font=small} % added 23-04-2019 Dennis Wittich

% My Definition
\usepackage{caption}
\usepackage{enumitem}
\usepackage{amsmath}
\usepackage{amssymb}
\usepackage{amsfonts}
\usepackage{pifont}
\newcommand{\CheckmarkBold}{\ding{51}} % チェックボックス
\newcommand{\XSolidBrush}{\ding{55}} % ばってん
\usepackage{ulem} % 打ち消し線
\usepackage{bm}
\usepackage{empheq}
% \usepackage{physics}
\usepackage{subcaption} 
\captionsetup{justification=raggedright}
\usepackage{cuted}
% \usepackage{colortbl}


\newcommand{\diff}[2]{\frac{\partial #1}{\partial #2}}

\usepackage[style=apa, backend=biber]{biblatex}
\addbibresource{ra-gs.bib}
\usepackage{ragged2e} 
\usepackage{xurl} % URLを任意の位置で改行可能にする
\AtNextBibliography{\sloppy} % 参考文献リスト内でのみ、改行ルールを少し緩くする \sloppy コマンドを適用
% \AtNextBibliography{\RaggedRight} % sloppyより強力な改行命令
% \usepackage[dvipdfmx, colorlinks=true]{hyperref} 
\usepackage[colorlinks=true]{hyperref} 
\usepackage[capitalize]{cleveref} % hyperrefの後ろ


\everymath{\displaystyle} % 分数を縦に潰さない


% 本文
\begin{document}

\title{RA-GS : Refractive-Aware Gaussian Splatting for Geometrically Accurate and Photorealistic 3D Reconstruction of Bathymetry from UAV-based Imagery}
\date{2025}

\author{
 Taiki Uno\textsuperscript{1}, 
 Sohei Kobayashi\textsuperscript{2}
}
\address{
	\textsuperscript{1 }Kyoto University, Graduate School of Engineering, Kyoto, Japan - uno.taiki.65r@st.kyoto-u.ac.jp\\
  \textsuperscript{2 }Kyoto University, Disaster Prevention Research Institute, Uji, Japan - kobayashi.sohei.7a@kyoto-u.ac.jp\\
  }


\abstract{
  The 3D reconstruction of geometry in shallow water has a wide range of applications, such as monitoring bed morphology changes and conducting hazard simulations. 
In addition, the detailed texture of the reconstructed data is useful for bed surface classification and aquatic habitat quantification. 
In this context, bathymetric surveying using photogrammetry from aerial images captured by Unmanned Aerial Vehicles (UAVs) is gaining attention as a particularly efficient method. 
However, it is fundamentally challenged by light refraction at the air-water interface, which invalidates the geometric principles of photogrammetry. 
Existing methods either rely on iterative post-processing or employ deep learning models that lack physical guarantees and explainability. 
We solve this challenge by introducing a refraction-aware 3D Gaussian Splatting framework that incorporates an optically accurate model of two-media refraction directly into the reconstruction pipeline. 
Our key innovation is a differentiable coordinate transformation that analytically models light refraction, mapping 3D Gaussians from their true underwater positions to their apparent space for each aerial view. 
This enables unified optimization, simultaneously solving for dense scene geometry and detailed appearance while maintaining the efficiency of standard 3D Gaussian Splatting. 
We evaluated our method on a simulated UAV dataset of a riverbed, rendered with physically-based ray tracing to isolate refractive effects from other optical phenomena. 
Our approach achieved a geometric F1-score of 96\% (with a 10 cm error threshold at a depth scale of 10 m). 
Furthermore, in novel view synthesis, we obtained photorealistic views with a Peak Signal-to-Noise Ratio (PSNR) of 25.9 dB and a Structural Similarity Index Measure (SSIM) of 0.93. 
By creating 3D models that are both photorealistic in appearance and dense and geometrically precise in structure, our method addresses a key challenge in aquatic remote sensing from aerial imagery. 
This work enables cost-effective, high-frequency monitoring of riverbeds, lakeshores, and seashores under calm surface conditions.
}

\keywords{
  Bathymetry, Gaussian Splatting, 3D Reconstruction, Two-Media Photogrammetry, Refraction Correction
}

\maketitle

\section{Introduction}\label{INTRODUCTION}
\sloppy

\begin{figure*}[htbp]
  \centering
  \captionsetup{justification=raggedright}
  \includegraphics[width=0.9\linewidth]{figure/Intro/overview_task2.pdf}
  \caption{Overview of our proposed method for photogrammetric bathymetry. 
  Our Refractive-Aware Gaussian Splatting takes multi-view images distorted by refraction (top right) as input. 
  By explicitly modeling refraction within the rendering pipeline, it reconstructs a geometrically accurate and refraction-free 3D scene (bottom left).}
  \label{fig:photogrammetric-bathymetry}
\end{figure*}

Bathymetry is critically important.
Accurate topographic information serves as fundamental data for river management, including flood prediction and monitoring geomorphological changes, and it also has potential applications in marine environments as well.
However, sonar-based hydrographic surveying, which is widely used from vessels, faces significant limitations in shallow waters due to the risk of grounding and equipment damage, and narrow and inefficient survey width.  
More recently, Airborne Light Detection and Ranging Bathymetry (ALB) has been employed, but both the equipment and data acquisition costs remain prohibitively high.  
On land, UAV-based photogrammetry has become a widely adopted technique, offering an efficient, inexpensive, safe, and accessible solution.  
Nonetheless, when applied to underwater targets from aerial platforms, photogrammetry encounters fundamental challenges.  
These include specular reflections from the water surface, distortions by waves, and light scattering and attenuation within the water medium.  
Among these factors, optical refraction is particularly detrimental, as it fundamentally breaks the assumption that light travels in a straight line, which is the foundation of photogrammetry.  

A variety of methods have been proposed to mitigate refraction effects, but most rely on post-hoc or iterative corrections applied to the outputs of conventional photogrammetric pipelines.  
Since they lack explicit physical modeling of refraction, such approaches fail to generalize to scenarios involving oblique or close-range imaging, leaving the core problem unresolved.  
Recently, refraction-aware structure-from-motion (R-SfM) methods have emerged \parencite{Makris2024_refractive-aware-sfm}, explicitly incorporating refraction into the SfM pipeline.  
Although promising, these methods typically yield only sparse point clouds and rely on deep learning–based densification techniques trained on ALB-derived datasets \parencite{Agrafiotis2019ISPRS_SVM-UAV-Bathymetry}.  
Given the scarcity of high-quality training data, applying these approaches across diverse regions remains infeasible.  

Concurrently, 3D Gaussian Splatting (3DGS) has recently gained significant attention as an emerging paradigm for 3D reconstruction \parencite{Kerbl2023ToG_3DGS}.  
Building on this advancement, we propose a novel foundation for photogrammetric bathymetry: Refractive-Aware Gaussian Splatting, which directly integrates an optically accurate two-media refraction model into the Gaussian Splatting pipeline.  
Our methods not only enables the recovery of geometrically accurate bathymetry but also provides photorealistic renderings for a clear appearance with refraction artifacts effectively removed.  
The key idea is to incorporate a differentiable parameter transformation that adjusts each Gaussian’s parameters with respect to the camera ray according to appearance depth, embedding this directly into the optimization process of Gaussian Splatting.  

Under the assumption of a calm, planar water surface without reflection and attenuation underwater, 
Our method reconstructs accurate 3D Gaussians free of refraction effect from multi-view images with known camera poses and refractive surface position, initialized with only a coarse prior of the depth of the floor.  
On multiple synthetic CG datasets, our method achieved image quality metrics exceeding PSNR 25 and demonstrated shape reconstruction with F1-scores above 90\% after 3DGS to point cloud conversion.  
This establishes a foundation for a next generation of photogrammetric bathymetry.  

\newpage


\section{Literature Review}\label{RELATED}
\sloppy

Bathymetry in shallow water area has been explored through a variety of methodologies \parencite{He2024_survey-shallow-bathymetry}.
Traditional fields measurements methods with bathymetric rods or hammers, and global positioning systems (GPS) are extremely time-consuming and dangerous.
Shipborne sonar systems are a representative method of bathymetry, however, the operation is restricted due to its narrow scanning swath. % 引用をつける
Therefore, remote sensing that reconstructs underwater morphology from data observed in the air is valuable in this field.
As remote sensing for shallow bathymetry, platforms equipped with sensors are generally classified into satellite, airborne, and, UAV-based systems.
The sensors employed range from RGB cameras and multi/hyper-spectral cameras, to LiDAR and even Synthetic Aperture Radar (SAR).
As an efficient, inexpensive, safe, and accessible solution, Our work focuses on methods based on UAV-based RGB imagery.
% radiometric な手法に関して言及すべき?

To reconstruct 3D information from UAV imagery, previous works have employed geometric approaches such as Structure-from-Motion (SfM) \parencite{schoenberger2016_colmap} or Multi-View-Stereo (MVS) \parencite{Furukawa2010_PatchMVS,Furukawa2015_MVS},
which have been developed in the computer vision community.
In a typical incremental SfM pipeline, keypoints are firstly detected and matched across frames using feature detectors and descriptors such as SIFT \parencite{Lowe2004_SIFT}.
Fundamental matrix $F$ between two images is then calculated, commonly via eight-points algorithm \parencite{Hartley1997_8PointAlgorithm} combined with random sample consensus (RANSAC) \parencite{Fischler1987_RANSAC}, which lead to camera pose recovery through singular value decomposition.
New camera poses are iteratively registered via Perspective-n-Point algorithms, which align estimated 3D points with 2D features in new frames.
Triangulation is subsequently applied to obtain additional 3D points from feature correspondences, and bundle adjustment (BA) \parencite{Triggs2000_BA} is finally performed to minimize reprojection error, refining both camera parameters and 3D points. 
MVS builds upon the calibrated cameras estimated from SfM, computing dense per-pixel depth and normal maps across views, which are then fused into a dense 3D point cloud or mesh.

This entire pipeline assumes straight-line traveling of light, while in reality, refraction at water-air interface significantly degrades both SfM and MVS reconstructions.
This refractive distortion represents the key challenge in photogrammetric bathymetry. 
% Previous studies (and also our method) have attempted to address this issue under simplifying assumptions, such as high water clarity, minimal surface waves, calm water conditions, and textured visible beds.
\cite{Woodget2014_PhotograBathy-multiply-n} demonstrated the potential of photogrammetric bathymetry from UAV imagery, applying a simple refraction correction by multiplying refractive index to SfM software outputs.
\cite{Dietrich2016_multi-angle-correction} proposed multi-angle refraction correction based on the pixel ray angle and estimating 3D points, which consider the view-angle dependence of each apparent points. 
However, this method only corrects for vertical displacement, ignoring horizontal distortions.
\cite{Makris2024_refractive-aware-sfm} successfully integrated refraction modeling directly into the SfM pipeline, avoiding post-hoc or iterative correction. 
This R-SfM method enables highly accurate camera pose estimation and point clouds generation.
They combine Deep Learning method \parencite{Alevizos2022_DL-shallow-bathymetry} to compensate the sparseness of SfM output, however, Deep Learning based method needs extensive training datasets and remain highly site-dependent.















% \subsection{Bathymetry in shallow water}\label{BATHYMETRY-SHALLOW}



% \begin{itemize}
%   \item 船。コントロール難しい。ラジコンだと非効率
%   \item 人。論外
%   \item ALB。いい感じやけど高い。
%   \item マルチスペクトル。今、勉強
% \end{itemize}

% \subsection{Photogrammetric Bathymetry}\label{PHOTOGRAMETRIC-BATHYMETRY}
% 歴史は古いよ。\par
% まず、SfM、MVSパイプラインの結果に屈折率をかけたりしたよ。\par
% カメラとの位置関係などを考慮して補正する量を決めたよ。\par
% 画像自体を処理していく考え方が生まれたよ。\par
% 鶏と玉子の関係からイテラティブな処理になったよ。\par
% Refraction-Aware SfMでは、パイプラインに明示的に屈折を組み込んだよ。(自分もその問題に取り組んでいたよ)\par
% 多分だけど、Appearance Depthを用いていないから、光線状の交点を求める必要があってバンドル調整できないのかな?\par
% Machine Learningを使用する方法も提案されているよ。\par
% 明示的に、end-to-endで、屈折のないシーンを推定できて、出力が密だったら嬉しいよ。\par

% \subsection{Radiance Field}\label{RADIANCE-FIELD}
% NeRFに始まる、Photometricな3D Reconstructionだよ。\par
% 画像から、幾何的な情報ではなくPhotometricな情報と3Dモデルのレンダリングを一致させるように最適化するよ。\par

% SonarベースのBathymetric Surveyが難しいShallow Water Areaにおいて、様々な手法が提案されている。
% (Survey論文を参考に)
% Satellite Image と UAV画像とを比較すると、UAVの方が個人や研究室単位で、頻度高く、自由度を持って、かつ解像度を上げたデータ取得が可能である。
% 水中での光スペクトルの減衰率を基にしたRadiometricな手法とGeometricな手法がある。
% 測量という観点から、

\section{Preliminaries}\label{PRELIMINALIES}
\sloppy

\subsection{3D Gaussian Splatting}\label{3DGS}

3D Gaussian Splatting (3DGS) is a recent method that has achieved state-of-the-art results in novel view synthesis \parencite{Kerbl2023ToG_3DGS}.
It is distinguished by its remarkable capability for high-fidelity 3D scene capturing with photorealism, rapid training times, and real-time rendering.
As an explicit 3D representation, 3DGS has been successfully applied to a wide range of tasks, including Visual-SLAM \parencite{Yan2024CVPR_GS-SLAM,Zheng2025CVPR_WildGS-SLAM,Matsuki2024CVPR_GaussianSplattingSLAM}, human avatar creation \parencite{Moreau2024CVPR_HumanGaussianSplatting,Shao2024CVPR_GaussianAvatar}, and feedforward 3D reconstruction \parencite{Chen2024ECCV_MVSplatting}.
Its potential further extends to various real-world applications such as digital surface model (DSM) generation from satellite imagery \parencite{Aira2025CVPR_EOGS}, UAV-based surveying, autonomous driving, and underwater 3D reconstruction \parencite{li20243DV_watersplatting}.


The 3DGS pipeline consists of two primary stages: a forward pass for rendering and a backward pass for optimization. 
In the forward pass, a collection of 3D Gaussians is rasterized to synthesize an image.
Each Gaussian is defined by a set of optimizable parameters: a center position $\bm{p} \in \mathbb{R}^3$, an opacity $\alpha \in [0, 1]$, view-dependent color coefficients represented by Spherical Harmonics (SH) $\bm{c}(\bm{p}, \bm{t}_i) \in \mathbb{R}^{3}$, and a 3D covariance matrix $\bm{\Sigma}^{\mathrm{3D}} \in \mathbb{R}^{3 \times 3}$. 
The covariance matrix $\bm{\Sigma}^{\mathrm{3D}}$ is composed by a scaling diagonal matrix $\bm{S} \in \mathbb{R}^{3 \times 3}$ composed of a scaling vector $\bm{s} \in \mathbb{R}^3$ and a rotation quaternion (represented as a rotation matrix $\bm{R} \in SO(3)$) as:
\begin{equation}\label{eq:3dgs-sigma_RssR}
  \bm{\Sigma}^{\mathrm{3D}} = \bm{R} \bm{s} \bm{s}^\top \bm{R}^\top
\end{equation}
The corresponding unnormalized Gaussian distribution function for a point $\bm{x} \in \mathbb{R}^3$ is given by:
\begin{equation}\label{eq:3dgs-G(x)}
  G(\bm{x}) = 
    \exp \left( -\frac{1}{2} (\bm{x} - \bm{p})^T \left({\bm{\Sigma}^{3D}}\right)^{-1} (\bm{x} - \bm{p}) \right)
\end{equation}
During the forward pass, to render an image from a given camera view, these Gaussians are first transformed from world to camera coordinates using the extrinsic matrix $[\bm{W} | \bm{t}]$ where $\bm{R}_\mathrm{view} \in \mathbb{R}^{3 \times 3}$ is a viewing rotation matrix and  $\bm{t} \in \mathbb{R}^3$ is a translation vector. 
The position of a Gaussian center $\bm{p}$ and 3D covariance matrix are updated as:
\begin{align*}
  &\bm{p}_\mathrm{cam} = \bm{R}_\mathrm{view}\bm{p} + \bm{t} \\
  &\bm{\Sigma}^{3D}_\mathrm{cam} = \bm{R}_\mathrm{view} \bm{\Sigma}^{3D} \bm{R}_\mathrm{view}^\top
\end{align*} 
Following the projection method from \parencite{Zwicker2001_EWA-volume-splatting}, the 3D covariance in camera space, $\Sigma_{cam}^{3D}$ ,is then projected onto the 2D image plane. This is achieved using the Jacobian $\bm{J}$ of the affine approximation of the perspective projection, yielding a 2D covariance matrix $\bm{\Sigma}^{\mathrm{2D}}$:
\begin{equation}\label{eq:3dgs-affine-projection}
\Sigma^\mathrm{2D} = \bm{J} \Sigma^{\mathrm{3D}}_\mathrm{cam} \bm{J}^\top
\end{equation}
The final RGB value $\bm{\Gamma} \in \mathbb{R}^{3}$ for each pixel is rendered by alpha-blending the projected Gaussians. 
The set of Gaussians that overlap with the pixel are first sorted front-to-back based on their depth, and the view-dependent color is then accumulated as:
\begin{align}\label{eq:3dgs-alpha-blending}
  \bm{\Gamma}(\bm{x}) = \sum_{k=1}^{K} \bm{c}_k \alpha^{\text{pixel}}_k \prod_{j=1}^{k-1} (1-\alpha^{\text{pixel}}_j) \\
  \quad \text{where} \quad \alpha^{\text{pixel}}_k = \alpha_k G_k^{\mathrm{2D}} \notag
\end{align}
here $k$ is the ordered set of Gaussians overlapping the pixel.

During the backward pass, the optimization minimizes a photometric loss, which is a weighted sum of a $\mathcal{L}_1 (\Gamma, \Gamma_{gt})$ Loss and a D-SSIM loss $\mathcal{L}_\mathrm{D-SSIM} (\Gamma, \Gamma_{gt})$ \parencite{Zhou2004_SSIM}:
\begin{equation}\label{eq:loss-function}
  \mathcal{L} = (1 - \lambda) \mathcal{L}1 + \lambda \mathcal{L}_\mathrm{D-SSIM}
\end{equation}
Thus, the optimization problem is formulated as follows:
\begin{equation}\label{eq:optimization}
  \underset{p, R, s, c, \alpha}{\textrm{argmin}} \quad \mathcal{L} = \mathcal{L}(\bm{\Gamma}, \bm{\Gamma_{gt}}) 
\end{equation}
These formulations ensure that the entire pipeline is fully differentiable, allowing the parameters $\{\bm{p}, \bm{R}, \bm{s}, \bm{c}, \alpha\}$ to be optimized via gradient descent using the Adam optimizer \parencite{Adam}.

The collection of 3D Gaussians obtained through this process captures the 3D scene with high fidelity. 
However, this entire pipeline relies on the pinhole camera model and perspective projection, which fundamentally assume that light travels in a straight line. 
This assumption is violated in environments with multi-medium interfaces, such as where refraction at the air-water boundary causes severe geometric inconsistency, leading to a failure in the reconstruction.

Despite this limitation, 3DGS is uniquely suited for addressing this challenge compared to implicit representations like NeRF. 
This is because the explicit nature of the Gaussian primitives that is highly open to direct physical modeling. 
It allows us to mathematically formulate and apply the laws of refraction directly to the geometric parameters of the scene representation itself.

%%%%%%%%%



% from collection of each Gaussian $\{G_k(\bm{x}) | k=1, \cdots, K\}$

% As an example, the Jacobian for the Gaussian center for the Loss function is as follows:
% \begin{align}
%   \diff{\mathcal{L}}{p_{k}}
%     &= \sum_{\bm{x} \in \mathrm{image}}
%     \diff{\mathcal{L}}{\Gamma}
%     \diff{\Gamma}{G(\bm{x})_k^{\mathrm{2D}}}
%     \diff{G(\bm{x})_k^{\mathrm{2D}}}{u_{i}}
    
%     \diff{u_{i}}{p_{c,i}}
    
%     \diff{p_{c,i}}{p_{w,i}} \\
    
%     &+  \sum_{\bm{x} \in \mathrm{image}}{
    
%     \diff{\mathcal{L}}{\Gamma_j}
    
%     \diff{\Gamma_j}{c_i}
    
%     \diff{c_i}{p_{w,i}}} \\
    
%     &+  \sum_{\bm{x} \in \mathrm{image}}{
    
%     \diff{\mathcal{L}}{\Gamma_j}
    
%     \diff{\Gamma_j}{\alpha_{ij}^{\prime}}
    
%     \diff{\alpha_{ij}^{\prime}}{\sigma_i^{\prime}}
    
%     \diff{\sigma_i^{\prime}}{p_{c,i}}
    
%     \diff{p_{c,i}}{p_{w,i}}
    
%     }
    
% \end{align}
% where $\bm{u}_k$ is the projected point of Gaussian mean in image coordinate.

\newpage

\section{Methodology}\label{METHODOLOGY}
\sloppy

\begin{figure}[htbp]
  \centering
  \captionsetup{justification=raggedright}
  \includegraphics[width=0.9\linewidth]{figure/method/position/overview.pdf}
  \caption{
  Overview of Refractive-aware Gaussian Splatting. 
  This figure illustrates our proposed pipeline. 
  A 3D scene, represented by 3D Gaussians, is transformed into an "apparent space" for each camera view by applying a differentiable refraction correction, which consists of position, scale, and opacity correction. 
  This transformation models the physical effects of light refraction at the air-water interface. 
  The resulting Gaussians are rendered to produce an image that includes refractive distortions.
  The core of our method is a fully differentiable process. A loss is computed by comparing the rendered image with the input source image (which also contains refraction). This loss is then backpropagated to optimize the parameters of the original 3D Gaussians. Through this optimization, the model learns a refraction-free 3D representation of the scene, enabling the generation of both photorealistic, undistorted novel views and a geometrically accurate 3D model.
  }
  \label{fig:overview}
\end{figure}

As illustrated in \cref{fig:overview}, our method explicitly models refraction by applying a corrective transformation to the parameters of each Gaussian primitive. 
This transformation accounts for the refractive effect of bending light rays at the air-water interface.
This refractive transformation is composed of three components:
(a) position correction, (b)scale correction, and (c)opacity correction.


\subsection{Position Correction}\label{center-correction}

We assume a Cartesian coordinate system where the camera is positioned on the positive z-axis at $(0, 0, H)$, and the water surface corresponds to the xy-plane ($z=0$). 
Following prior work, we assume the water surface is flat and calm, resulting in a constant surface normal $\mathbf{n} = (0, 0, 1)$. A 3D Gaussian primitive located underwater has a center position $\bm{p} = (x, y, z)$, where $z < 0$.  % [TODO] 既存研究

To simplify the geometry, we project the 3D position onto the 2D \textit{rz-plane} using the radial distance $r = \sqrt{x^2 + y^2}$. 
The apparent position of the Gaussian's center, after accounting for refraction, is denoted as $\bm{p'} = (x', y', z')$, which projects to $(r', z')$ on this plane.

From Snell's law and geometric optics, it follows that the apparent position of a submerged point, as seen from a fixed camera center, is uniquely determined\parencite{nassar1994_ApparentDepth,Missailidis2025_apparentDepth-leading}.
The ray from the apparent position $\bm{p'}$ to the camera intersects the water surface at a point we define as $(s, 0)$ on the \textit{rz-plane}, where $0 \leq s < r$.
Let $\theta_r$ be the angle of incidence (in water) and $\theta_i$ be the angle of refraction (in air), both measured with respect to the surface normal. 
Let $n$ be the refractive index of water, and we assume the refractive index of air is 1. 
Snell's Law gives:
\begin{equation}\label{eq:snell}
  \sin \theta_i = n \sin \theta_r
\end{equation}
From the geometry of the system, these angles can be related to the positions, leading to a quartic equation in $s$
(triangle similarity between $\triangle OAI$ and $\triangle IPC$):
\begin{equation}\label{eq:quartic-s}
\begin{split}
  (1-n^2)s^4 &+ 2(n^2-1)rs^3 \\
             &+ ((1-n^2)r^2 + H^2 - n^2 z^2)s^2 \\ % Note: Original had H^2 and z^2 swapped in position, check your derivation. Assuming H is camera height and z is object depth.
             &+ 2n^2z^2rs - n^2z^2r^2 = 0
\end{split}
\end{equation}
This equation can be solved numerically for $s$, for instance, using Newton's method or directly solved by Ferrari's method. 
To find the apparent position $(r', z')$, we analyze the differential change in ray position. 
From the geometry of triangles formed by the rays and positions, we can establish the following relationships.
( $\triangle IPC$, $\triangle IP'C'$, and $\triangle I'P'C'$):
\begin{equation}\label{eq:r-s}
  r - s = -z \tan \theta_r
\end{equation}
\begin{equation}\label{eq:triangle-IP'C'}
  r' - s = z' \tan \theta_i
\end{equation}
\begin{align}
  -z' \tan \left(\theta_i + \Delta \theta_i \right) & = (r' - s) + (- \Delta s) \notag \\
  -z' \left( \sin \theta_i + \Delta \theta_i \cos \theta_i  \right) & = (r' -s - \Delta s) \left( \cos \theta_i - \Delta \theta_i \sin \theta_i \right) \label{eq:triangle-I'P'C'}
\end{align}
With the derivative of Snell's law from \cref{eq:snell}:
\begin{equation}\label{eq:dtheta_r/dtheta_i}
  \frac{d \theta_r}{d \theta_i} = \frac{1}{n} \frac{\cos \theta_i}{\cos \theta_r}
\end{equation}
By differentiating these geometric relationships \cref{eq:triangle-I'P'C',eq:triangle-IP'C',eq:r-s} and combining them with equations of Snell's law \cref{eq:snell,eq:dtheta_r/dtheta_i},
we can represent the apparent position $(r', z')$ with $\theta_i$ and $\theta_r$.
\begin{subequations}
  \begin{empheq}[left=\empheqlbrace]{align}
      r' &= r + (n^2 -1) \cdot z \cdot \tan^3 \theta_r \label{eq:r'} \\
      z' &= \frac{1}{n} \cdot \frac{{\cos \theta_i}^3}{{\cos \theta_r}^3} \cdot z \label{eq:z'}
  \end{empheq}
\end{subequations}
This is the center position correction transformation. 
Since $\theta_i$ and $\theta_r$ are determined by the camera position and the Gaussian's mean position $\bm{p}$, the apparent position $\bm{p'}$ is uniquely determined. 
The transformation is applied by first calculating $r$ and then finding the new coordinates $x' = x \frac{r'}{r}$ and $y' = y \frac{r'}{r}$.
Note that since $z<0$, the term $(n^2 -1) \cdot z \cdot \tan^3 \theta_r$ is negative, ensuring $r' < r$. 
Similarly, since $\theta_i > \theta_r$ for $n>1$, we have $\cos \theta_i < \cos \theta_r$, which ensures $|z'| < |z|$. 
This confirms that the apparent Gaussian $G'(\bm{p'})$ always appears shallower and closer to the central axis than the actual Gaussian $G(\bm{p})$.
In the special case where the Gaussian is directly below the camera ($r=0$, thus $\theta_i = \theta_r = 0$), the equations simplify to $r'=r$ and $z' = z/n$. 
This special case, where apparent depth is the true depth divided by $n$, is a common approximation used in prior work (e.g., \cite{Woodget2014_PhotograBathy-multiply-n}). 
However, this approximation is inaccurate for most rays in a perspective projection which are off-nadir.

For use in Gaussian Splatting, this transformation must be differentiable. 
We therefore require the Jacobian of the transformation:
\begin{equation}\label{eq:Jacobian}
  \begin{split}
    J_{app} = 
    \frac{\partial \bm{p'}}{\partial \bm{p}} &= 
    \begin{bmatrix} 
      \frac{\partial x'}{\partial x} & \frac{\partial x'}{\partial y} & \frac{\partial x'}{\partial z} \\
      \frac{\partial y'}{\partial x} & \frac{\partial y'}{\partial y} & \frac{\partial y'}{\partial z} \\ 
      \frac{\partial z'}{\partial x} & \frac{\partial z'}{\partial y} & \frac{\partial z'}{\partial z} 
    \end{bmatrix} \\
  \end{split}
\end{equation}
While this Jacobian can be derived analytically, we compute it using numerical differentiation in our implementation for simplicity. Specifically, we use finite differences; for example, the partial derivative ${\partial x'} / {\partial x}$ is approximated as $(p'_x(x+\Delta x) - p'_x(x-\Delta x)) / (2\Delta x)$.

This correction is the key proposal of our paper, which models the effect of refraction through a direct analytical transformation of the Gaussian means. 
This enables the reconstruction of an accurate, refraction-free 3D scene from posed images of an underwater environment. 
The only required inputs are the refractive index of the water and the location of the water surface relative to the cameras.


\begin{figure}[htbp]
  \centering
  \captionsetup{justification=raggedright,singlelinecheck=false}

  % 左
  \begin{minipage}[t]{0.48\linewidth}
    \centering
    \includegraphics[width=\linewidth]{figure/method/rz.png}
    \caption{Illustration of the apparent position $P^{\prime}$ versus the true position $P$ in the \textit{rz}-plane.
    Even though the actual object is located at the blue Gaussian position $P$, it appears shifted toward the red Gaussian position $P'$ when observed through the refractive interface.
    }
    \label{fig:rz}
  \end{minipage}
  \hfill
  % 右
  \begin{minipage}[t]{0.48\linewidth}
    \centering
    \includegraphics[width=\linewidth]{figure/method/space_compression.png}
    \caption{Space compression from the camera view. 
    The Space is distorted by the camera position.
    Each colored grid means the corresponding compressed space of entire grid in camera coordinates (shown as black.)}
    \label{fig:space_compression_by_view}
  \end{minipage}
\end{figure}



\subsection{Scale Correction}\label{scale-correction}

The transformation of the Gaussians center position $\bm{p} \to \bm{p'}$, which models the core effect of refraction, distorts the Cartesian coordinates system of camera coordinates.
This space is named \textit{the apparent space}, where the axes are no longer orthogonal as illustrated in \cref{fig:space_compression_by_view}.
This distortion is particularly pronounced for large angles of incidence (in the air), causing the space to be significantly compressed. 
Without a corresponding correction to the scale parameter $\bm{s}$, individual Gaussians are perceived as being both densely packed and dilated. 
This leads to severe visual artifacts, such as blurring and unrealistic expansion of surfaces, as shown in \cref{fig:artifact-w/o-scale-correction}.

\begin{figure}[htbp]
  \centering
    \includegraphics[width=0.8\linewidth]{figure/method/artifact-wo-scale-correction.png}
    \caption{Artifact without scale correction. \\ 
    \textbf{Left}: The image is rendered image of 3DGS model trained by Ground Truth (GT) images without refractive plane.
    \textbf{Right}: The image is rendered of same model using center position correction, but without scale correction.
    The scene in area bounded by the red is expanded, where the ray of incident angle is larger.}\label{fig:artifact-w/o-scale-correction}
\end{figure}

To mitigate these artifacts, we propose to downscale each Gaussian according to the local space compression ratio. 
We quantify this compression by analyzing the transformation of the Cartesian basis vectors. 
For instance, the basis vector \(\mathbf{e}_x = (1,0,0)^\top\) is to be mapped by given by the first column of $J_{app}$:
\begin{equation}
    \mathbf{e}_{x}' = J_{app} \mathbf{e}_x = 
    \begin{bmatrix}
    \frac{\partial x'}{\partial x} \\
    \frac{\partial y'}{\partial x} \\
    \frac{\partial z'}{\partial x}
    \end{bmatrix}.
\end{equation}

The local scaling factor along the \(x\)-axis is quantified by the norm of this vector:
\begin{equation}
    s_x = \|\mathbf{e}_{x}'\| = \sqrt{\left(\frac{\partial x'}{\partial x}\right)^2 + \left(\frac{\partial y'}{\partial x}\right)^2 + \left(\frac{\partial z'}{\partial x}\right)^2}.
\end{equation}
Similarly, we can compute the scaling factor $s_y$ and $s_z$ along the $y$-axis and $z$-axis.

It seems better in order to calculate the volume compression ratio $V$ by applying to a determinant of \cref{eq:Jacobian} $|\det J_{app}|$.
However, this value overestimates downscaling especially the ray of incident angle is larger, which makes Gaussians too small.

While one might consider using the determinant of the Jacobian $|\det J_{app}|$ as a direct measure of the volumetric compression, we empirically found that this approach overestimates the required downscaling, particularly for large incidence angles, resulting in overly diminished Gaussians.
Instead, we define an isotropic scale correction factor $S$ derived from the geometric mean of the directional scaling factors. 
This factor is computed as the cubic root of the product of the individual scaling factors:
\begin{equation}
  \label{eq:volume-compression-ratio}
  V = s_x \cdot s_y \cdot s_z
\end{equation}
This formulation provides a stable measure of the average spatial compression that is robust to the extreme shear in the transformation.
While it intentionally disregards the shear components, it is precisely this property that makes it effective for isotropically rescaling the Gaussians, preventing the visual artifacts without eliminating the primitives themselves.


\begin{equation}\label{scale-correction-factor}
  S = V ^ {1/3}
\end{equation}
We then apply this factor to the original scale vector  $\bm{s}$ to obtain the corrected scale $\bm{s'}$
\begin{equation}
\label{eq:scale-correction-edge}  
  \bm{s'} = S \bm{s}
\end{equation}
We apply this correction during the rendering phase, dynamically adjusting the scale of each Gaussian based on its position relative to the camera and the refractive interface.


% 1.  **"Why the geometric mean?":** A sharp reviewer will ask: "The product of norms `s_x s_y s_z` is not the true volume change, which is given by the determinant. Your method ignores the shear/non-orthogonality of the transformed basis vectors. Please justify why this simplification is valid or even preferable."
%     **Your Defense:** You need to be ready. Your answer should be something like: "We acknowledge that our formulation `V` does not represent the true geometric volume change. However, our goal is to find an effective isotropic scaling factor for the Gaussian primitive, which itself is symmetric. We found that applying an isotropic correction based on the geometric mean of the directional stretches yielded visually superior results by preventing the excessive and often anisotropic shrinkage caused by using the determinant, which is more sensitive to shear."
% 2.  **"What are the limitations?":** "Does this method work for curved refractive surfaces or only planar ones? What about materials with varying indices of refraction?"
%     **Your Defense:** Be upfront about the scope of your work in the paper. State clearly that you've validated it for planar surfaces and that extending it to more complex, spatially-varying scenarios is a promising direction for future work. Never claim your method solves everything. Honesty about limitations builds trust with reviewers.

\subsection{Opacity Correction}\label{sec:opacity-correction}

Reconstructing scenes observed through a refractive interface introduces a characteristic radiometric attenuation.
As illustrated in \cref{fig:refraction-makes-scene-darker}, a scene viewed through water appears substantially darker due to etendue conservation \cite{etendue}. 
For a light bundle traveling from a medium with refractive index $n$ into air, the radiance is scaled as
\begin{equation}
L_{\mathrm{air}} = \frac{1}{n^{2}} L_{\mathrm{water}}
\end{equation}
This reduction in observed radiance poses a challenge for 3D Gaussian Splatting. 
When optimizing directly against the darker target images, the solution often collapses into a configuration with artificially low opacity.
Such semi-transparent Gaussians explain the low observed brightness but lead to geometrically ambiguous, blurry, and under-constrained reconstructions.

To counteract these phenomena, we introduce an opacity regularization inspired by the radiometric relationship above. 
Rather than allowing the optimizer to freely reduce opacity, we compensate the expected radiance loss by scaling the opacity of Gaussians located behind the refractive interface:
\begin{equation}
\alpha' = \frac{1}{n^{2}} \alpha
\end{equation}
This adjustment reduces the opacity budget available to each Gaussian, thereby discouraging trivial low-opacity solutions. 
As a result, the optimizer is driven to represent the scene using a denser and more spatially coherent set of primitives, improving both geometric and apparent fidelity.

It is important to emphasize that this strategy is not a physically exact model of refraction. 
Instead, it is a principled heuristic that stabilizes the optimization in refractive environments. 
Empirically, it yields consistently sharper geometry and avoids common artifacts such as floaters and oversmoothing.

\begin{figure}[htbp]
  \centering
  % 1枚目の画像
  \begin{minipage}[b]{0.40\linewidth}
    \centering
    \includegraphics[width=\linewidth]{figure/method/color/0050_gt_background.png}
  \end{minipage}
  % \hfill % 画像間のスペースを最大にする
  % 2枚目の画像
  \begin{minipage}[b]{0.40\linewidth}
    \centering
    \includegraphics[width=\linewidth]{figure/method/color/0050_refraction_background.png}
  \end{minipage}
  \caption{
    Comparison of path-traced renderings in Blender Cycles.
    The right image, viewed through a water surface, exhibits substantial radiance reduction compared to the left image rendered without refraction.
    Our opacity regularization mitigates the tendency of Gaussian Splatting to converge to low-opacity, blurry configurations under such conditions.}\label{fig:refraction-makes-scene-darker}
\end{figure}
\section{Experiments}\label{sec:EXPERIMENTS}

\subsection{Dataset}\label{sec:dataset}

\begin{figure}[!tbhp]
\centering
\includegraphics[width=0.7\linewidth]{figure/experiment/simulator.png}
\caption{Overview of the simulated data capture setup.}
\label{fig:simulator-settings}
\end{figure}

To isolate refractive effects from other optical phenomena such as surface reflection and underwater light attenuation, all experiments are conducted in a controlled simulation environment.
We acknowledge that these effects are non-negligible in real-world conditions, and addressing them remains future work.

A synthetic riverbed mesh is generated in Blender \cite{blender} by applying a displacement modifier to a gravel texture, and physically accurate images are rendered using the Cycles path tracer.
The water surface is assumed to be flat and free of specular reflections, with an average depth of 10 m.
Images were captured with 85\% side and forward overlap, combining nadir views with off-nadir shots at 20° and 40° inclination, rotating 10° around the target area for full coverage.
While multi-view observations benefit geometric consistency, larger angles amplify refractive distortion and apparent depth errors (\cref{fig:space_compression_by_view}).

The simulated camera follows a pinhole model with a 70° field of view and a resolution of 800 $\times$ 800 pixels.
Each dataset consists of 90 images, excluding novel-view test images.
An overview of the capture setup is shown in \cref{fig:simulator-settings}.

\subsection{Implementation}\label{sec:implementation}

We implement Refractive-Aware Gaussian Splatting by extending the open-source gsplat framework~\cite{ye2025_gsplat}.
The refractive parameter transformation is realized as a custom PyTorch~\cite{paszke2019pytorch} autograd function, integrated into gsplat’s differentiable rasterizer.
The quartic equation for center correction (\cref{sec:center-correction}) is solved via GPU-parallelized Newton–-Raphson iteration, introducing a modest overhead of up to $3 \times$  compared to the original gsplat implementation.

We adopt the adaptive density control and default hyperparameters from vanilla 3DGS and run optimization for 30,000 iterations.
To simplify gradient computation, the spherical harmonic degree is fixed to 0, making colors view-independent.
Feature-based SfM~\cite{schoenberger2016_colmap} frequently results in sparse point clouds on textureless riverbeds in real-world environments~\cite{Mandlburger2019_featureMatching-textureless}. Consequently, we initialize Gaussians as a planar point cloud at a depth of approximately 10 m.
All experiments are performed on a single NVIDIA RTX 3090 GPU.

\subsection{Geometry Extraction}\label{sec:geometry-extraction}

% \begin{figure}[htbp]
% \centering
% \includegraphics[width=\linewidth]{figure/experiment/flow_pc_extraction.png}
% \caption{Flow of point cloud extraction.}
% \label{fig:flow_pc_extraction}
% \end{figure}

\begin{figure}[!tbhp]
\centering
\includegraphics[width=\linewidth]{figure/experiment/extraction.png}
\caption{Geometry extraction pipeline.
Left: noisy point cloud from 3DGS.
Center: discretization and median-height computation.
Right: 2.5D height map and final surface-aligned point cloud.}
\label{fig:extract_surface_aligned_pc}
\end{figure}

To evaluate geometric accuracy, the optimized 3D Gaussians are converted into dense, surface-aligned point clouds suitable for bathymetric analysis.
While advanced extraction schemes exist (e.g., SuGaR~\cite{Gu2024CVPR_SuGaR}, 2DGS~\cite{Huang2024SIGGRAPH_2DGS}, TrimGS~\cite{fan2024_trimGS}), we employ a lightweight procedure for reproducibility.

Raw point clouds are generated using the 3DGS-to-PC converter~\cite{stuart20253dgstopcconvert3dgaussian}, which samples points proportionally to Gaussian volume.
To remove floating artifacts, local-plane outlier filtering is performed using CloudCompare~\cite{girardeau2016cloudcompare}.
The filtered cloud is discretized into a 2D grid on the horizontal plane; the median height of each cell defines a height map, from which we reconstruct a surface-aligned point cloud (\cref{fig:extract_surface_aligned_pc}).
Grid resolution and filtering parameters are adjusted per dataset to balance smoothness and detail preservation.
\section{Evaluation}\label{sec:EVALUATION}

The training images for our evaluation are those generated as described in \cref{sec:dataset}, which exhibit distortions caused by a refractive plane.

\subsection{Appearance Evaulation}\label{sec:appearance-evaluation}


\begin{table*}[htbp]
  \caption{Quantitative results of appearance results, compared to an ablation of proposed components. 
  Each model is trained from images with a refractive scene.}
  \label{tab:apparent-evaluation}
  \centering
  \small
  \scalebox{0.70}
  {
    \begin{tabular}{l|ccc|ccc|ccc|cc}
      
      & \multicolumn{3}{c|}{Correction Components} & \multicolumn{3}{c|}{Render w/ Refraction} & \multicolumn{3}{c|}{Render w/o Refraction} & \multicolumn{2}{c}{Stats}\\
      Method & Position & Scale & Opacity
      & $SSIM^\uparrow$   & $PSNR^\uparrow$    & $LPIPS^\downarrow$  
      & $SSIM^\uparrow$   & $PSNR^\uparrow$    & $LPIPS^\downarrow$  
      & \# Gaussians (K) & Time (min) \\
      \hline \hline 
      
      3DGS
      & \XSolidBrush & \XSolidBrush & \XSolidBrush
      % & 0.610 & 14.81 & 0.332 & 0.682 & 14.12 & 0.292 & 40.0 & 6.8 \\ 
      & - & - & - & 0.682 & 14.12 & 0.292 & 40.0 & 6.8 \\ 
      
      Position
      & \CheckmarkBold & \XSolidBrush & \XSolidBrush
      & 0.980 & 37.56 & 0.023 & 0.953 & 21.49 & 0.039 & 23.1 & 10.9 \\
      
      + Scale
      & \CheckmarkBold & \CheckmarkBold & \XSolidBrush
      & 0.964 & 37.21 & 0.023 & 0.936 & 22.97 & 0.043 & 28.0 & 21.0 \\

      + Opacity 
      & \CheckmarkBold & \XSolidBrush & \CheckmarkBold
      & 0.984 & 38.44 & 0.018 & 0.954 & 25.27 & 0.033 & 38.8 & 11.5 \\

      Ours
      & \CheckmarkBold & \CheckmarkBold & \CheckmarkBold
      & 0.981 & 38.42 & 0.017 & 0.933 & 25.97 & 0.046 & 41.3 & 21.9 \\

    \end{tabular}
  }
\end{table*}

Following the standards for Novel View Synthesis (NVS) evaluation, we assess the similarity between rendered images of our 3D model and the corresponding ground-truth images from unseen viewpoints. 
We employ three widely-used metrics: the Peak Signal-to-Noise Ratio (PSNR), which measures pixel-wise accuracy; 
the Structural Similarity Index Measure (SSIM) \parencite{Zhou2004_SSIM}, which considers local image structures; 
and the Learned Perceptual Image Patch Similarity (LPIPS) \parencite{Zhang2018CVPR_LPIPS}, which leverages deep features from a Convolutional Neural Network (CNN) to better approximate human perception.
Our method aims to reconstruct a 3DGS model that is free of refractive artifacts. 
To comprehensively evaluate this, we report metrics under two conditions, as shown in \cref{tab:apparent-evaluation}. 
First, we render the scene with the refractive plane ("Render w/ Refraction") and compare against the distorted ground-truth images. 
Second, we render the model without the refractive plane ("Render w/o Refraction") and compare against the undistorted ground-truth images.

The results demonstrate that correcting the Gaussian center positions plays the most crucial role, leading to a significant improvement across all metrics. 
For applications like bed surface classification, preserving structural details is paramount, making SSIM a key metric. 
Notably, our full method maintains an SSIM score above 0.93 even in the challenging "w/o Refraction" scenario.
However, the PSNR for our full method in this scenario is nearly 13 dB lower than in the "w/ Refraction" case. 
We attribute this to the physical incorrectness as discussed in \cref{sec:opacity-correction}.
Our opacity correction uniformly reduces the opacity of all Gaussians, causing background Gaussians to be composited rather than revealing a black background.
While the opacity regularization slightly contributes the increase of PSNR, the effect is insufficient to model the reduction of radiance at a refractive surface.
This results in a subtle change in luminance, a discrepancy that is heavily penalized by the pixel-wise PSNR metric. 
Therefore, we need the correction method that directly models the change of the luminance through water to air.

Furthermore, the inclusion of the scale correction slightly degrades the SSIM score. 
This is likely because our method isotropically corrects the Gaussian volumes regardless of their shape or orientation, which can cause a blurring effect that harms fine structural details in the non-refractive rendering. 
Finally, while the training time for our method is over three times that of the baseline 3DGS, yet it remains within a practical range. 
We anticipate that this could be significantly accelerated with a dedicated CUDA implementation.



\begin{figure}[htbp]
  \centering
  \captionsetup{justification=raggedright}
  \includegraphics[width=0.9\linewidth]{figure/evaluation/real-space/appearance_align.png}
  \caption{Visual comparisons between our method, Ground Truth, ablation of each correction. 
           We also apply center position correction with scale correction and opacity correction.}
  \label{fig:ablation-appearance}
\end{figure}






\subsection{Geometry Evalation}\label{sec:geometry-evaluation}

\begin{table*}[htbp]
  \caption{Quantitative results of geometry, compared to an ablation of proposed components. 
  Each model is trained from images with a refractive scene.
  We report the averaged chamfer distance, F1 score for each threshold (10 cm, 30 cm)}
  \label{tab:geometry-evaluation}
  \centering
  \small
  \scalebox{0.70}
  {
    \begin{tabular}{l||ccc|c|ccc|ccc}
      
      & \multicolumn{3}{c|}{Correction Components} & &  \multicolumn{3}{c|}{10cm} & \multicolumn{3}{c}{30cm}\\

      Method & Position & Scale & Opacity
      & $CD^\downarrow$ (m)  
      & $Precise^\uparrow$ (\%)   & $Recall^\uparrow$ (\%) & $F1^\uparrow$ (\%)
      & $Precise^\uparrow$ (\%)   & $Recall^\uparrow$ (\%) & $F1^\uparrow$ (\%) \\
      \hline \hline 
      
      3DGS
      & \XSolidBrush & \XSolidBrush & \XSolidBrush
      & 8.477 & 0.03  & 0.20  & 0.06  & 0.12  & 2.52   & 0.23  \\
      
      Position
      & \CheckmarkBold & \XSolidBrush & \XSolidBrush
      & 0.110 & 69.80 & 84.65 & 76.51 & 82.80 & 99.95  & 90.57 \\
      
      + Scale
      & \CheckmarkBold & \CheckmarkBold & \XSolidBrush
      & 0.033 & 81.10 & 91.06 & 85.79 & 93.03 & 99.97  & 96.37 \\

      + Opacity 
      & \CheckmarkBold & \XSolidBrush & \CheckmarkBold
      & 0.054 & 79.29 & 90.80 & 84.66 & 90.64 & 99.99  & 95.09 \\

      Ours
      & \CheckmarkBold & \CheckmarkBold & \CheckmarkBold
      & 0.011 & 91.56 & 96.58 & 94.00 & 98.26 & 100.00 & 99.12 \\


    \end{tabular}
  }
\end{table*}


Accurately reconstructing the true geometry of a terrain observed through a refractive interface poses a significant challenge. 
In this section, we quantitatively evaluate our method's geometric accuracy by comparing the extracted point cloud against the ground truth from the simulator.
We employ two standard metrics for evaluating similarity between point clouds: Chamfer Distance (CD) and F1 Score. 
The Chamfer Distance measures the average squared distance between nearest neighbors in two point clouds $S_{gt}$ and $S_{est}$.
It is defined symmetrically as:
\begin{align}\label{eq:CD}
  \text{CD}(S_{gt}, S_{est}) &= \frac{1}{|S_{gt}|} \sum_{x \in S_{gt}} \min_{y \in S_{est}} \|x-y\|^2_2 \notag  \\
  &+ \frac{1}{|S_{est}|} \sum_{y \in S_{est}} \min_{x \in S_{gt}} \|x-y\|^2_2
\end{align}
A lower CD value indicates a closer match between the point clouds.

The F1 score assesses the overall quality by balancing precision and recall against a distance threshold $\tau$.
\begin{itemize}
  \item Precision: the percentage of points in the estimated cloud $S_{est}$ that are within the threshold $\tau$ of the ground-truth cloud $S_{gt}$.
  \item Recall: the percentage of points in $S_{gt}$ that are within $\tau$ of $S_{est}$. 
\end{itemize}
  
The F1 score is the harmonic mean of Precision and Recall, defined as:
\begin{equation}\label{eq:F1}
  F1 = \frac{2 \cdot Precision \cdot Recall}{Precision + Recall}
\end{equation}


As shown in \cref{tab:geometry-evaluation}, the baseline 3DGS performs extremely poorly on the refractive scene, resulting in a very high Chamfer Distance (8.477 m) and near-zero F1 scores, as it fails to account for the physics of refraction.
Our ablation study demonstrates the incremental contribution of each proposed component. 
Introducing position correction yields the most substantial improvement, significantly reducing the CD to 0.110 m and boosting the 10 cm threshold F1 score to 76.51\%. 
Adding scale correction further refines the geometry, lowering the CD to 0.033 m and increasing the F1 score to 85.79\%. 
While adding scale correction and opacity correction each provides incremental improvements, the synergistic integration of all three components in our full method ("Ours") achieves the best performance.
This comprehensive approach achieves the lowest Chamfer Distance of 0.011 m and the highest F1 scores of 94.00\% (at 10 cm) and 99.12\% (at 30 cm). 
These results are achieved on a challenging dataset featuring water depths of up to 10 meters, highlighting the robustness of our method. 
This demonstrates that our combined correction strategy is crucial for achieving high-fidelity geometric reconstruction, complementing the appearance enhancements shown previously.


\begin{figure}[htbp]
  \centering
  % 1枚目の画像
  \begin{minipage}[b]{0.45\linewidth}
    \centering
    \includegraphics[width=\linewidth]{figure/evaluation/geometry/gt.png}
  \end{minipage}
  \hfill % 画像間のスペースを最大にする
  % 2枚目の画像
  \begin{minipage}[b]{0.45\linewidth}
    \centering
    \includegraphics[width=\linewidth]{figure/evaluation/geometry/wo.png}
  \end{minipage}
  \begin{minipage}[b]{0.45\linewidth}
    \centering
    \includegraphics[width=\linewidth]{figure/evaluation/geometry/w.png}
  \end{minipage}
  \caption{Qualitative results of geometry.
  Top-left: The ground truth geometry rendered in simulator.
  Top-right: The extracted geometry without correction.
  Bottom-center: The extracted geometry with all correction.
  Each image is rendered from an identical viewpoint and under the same lighting conditions. 
  The ground truth and the corrected geometry are very similar in both position and shape. 
  In contrast, the uncorrected geometry appears completely noisy and is incorrectly positioned at a shallower depth.
  }
  \label{fig:geometry-evaluation-quantitatively}
\end{figure}


% 完璧じゃなくてNoiseが残っていることを示す。
% (BlenderでNormalMapと普通のRendaring)


% % \section{CONCLUSION}\label{sec:conclusion}
\subsection{Limitations}\label{LIMITATIONS}
\sloppy

\subsection*{Checklist for Discussion}
Here is a sample checklist of limitations to address.

\newlist{checklist}{itemize}{1}
\setlist[checklist]{label=$\square$, leftmargin=*}

\begin{checklist}
	\item フレネルの法則で支配される反射や全反射の関係を考えていない。
	\item 波の影響を考えていない。Surfaceをアラインする正則化\parencite{hoge_sugar}などによって、波に対して眼瞼に検証が可能になるはず。
	\item 水の減衰の影響を考えていない。\parencite{hoge_seathru,hoge_seathru_gs}などによって、いい感じにできると思う。
\end{checklist}

実用化には、カメラポーズを正確に得る必要があり、SfMを用いる。
今回はシミュレーションを用い、正確なカメラ内部外部パラメータが既知であることを仮定し検証を進めたが、水面のあるSceneと同様に、光の直進性が崩壊するため、実環境で正確なSfMの結果を得ることは難しい。
SfM内で屈折を考慮する\parencite{Makris2024_refractive-aware-sfm}では、カメラのポーズは既知と仮定されているが、今回導入した\cref{center-correction}は、微分可能な座標変換に過ぎず、Bundle Ajustmentの過程に容易に組み込むことができる。
よって、今回の提案手法、を応用することによって、水面を写した画像から正確なカメラポーズ推定が可能になる可能性がある。
また、カメラ高度から水面までの高さも既知と仮定していた。
これは、実際の測量ではUAVの飛行高度設定によっておおよおそ推定、制御可能である。
水面からの相対的な高さをバンドル調整に据えることで、正確な水面の位置も推定可能になると考えている。

波の影響
本研究は、水面は完全にPlaneであると仮定したが、実際の環境ではそういった状況はほとんど生じない。
3DGSがAppearanceベースの最適化である以上、Geometry的な正則化が必須になるだろう。
\parencite{Huang2024SIGGRAPH_2DGS,Gu2024CVPR_SuGaR}などで提案されるGaussianが表面に沿って配置されるようにすることで、複数視点からの撮影で平均的に波の影響を緩和できると思う。
また、\cref{eq:loss-function}において、画素ごとの信頼度であるPSNRよりもSSIMを信頼するように$\lambda$の割合を増やすというアイデアもある。
LPIPSをロス関数に組み込んでもよいだろう。

% To achieve a more comprehensive underwater rendering, this radiance scaling must be integrated with other optical phenomena, such as wavelength-dependent attenuation (i.e., light absorption and scattering) and Fresnel reflections at the water's surface, which we plan to investigate in subsequent research.

水の濁り
3DGSはAppearanceをもとに、

Mesh Estractionを、SuGARなど、よりエレガントな手法を用いる。目的の人るがGeometryなら2DGSを使うべき。


% "The evaluation is performed on synthetic data. How does the method generalize to real-world underwater scenes?"
% 対策: もし実データでの実験がなければ、「実環境では、水の濁り、動的な浮遊物、不正確な水面推定など、さらなる課題が存在する」と正直に認めること。その上で、「本研究は、屈折という根源的な問題に対する初の解決策の一つを提示するものであり、将来的にこれらの課題に取り組むための重要な基盤となる」と、研究の位置づけと貢献を明確に主張しよう。

\section{Conclusion}\label{sec:conclusion}
\sloppy

We presented refractive-aware Gaussian Splatting, which accurately models effects of refraction by leveraging the radiance field representation and the optimization framework of 3D Gaussian Splatting\parencite{Kerbl2023ToG_3DGS}.
Our differentiable transformation into apparent appearance enables refraction-free 3D reconstruction from images of a refractive scene. 
Our experiments demonstrate superior reconstruction quality, achieving high SSIM scores when our corrected model is compared against the pristine, non-refractive ground-truth images.
Furthermore, the extracted 3D geometry attains an F1 Score of over 99\% and a Chamfer Distance of 0.011 m, confirming its high geometric accuracy.

\subsection{Limitations and Future Work}\label{sec:limitation}

Our study primarily focuses on refractive effects, making our method best suited for idealized environments. 
In real-world scenarios, however, several other physical phenomena must be considered. 
Our model does not account for surface reflections, underwater light attenuation (i.e., absorption and scattering), or the Fresnel effect. 
This is a notable limitation, as our radiometric loss is minimized against observed images that inherently capture these phenomena.

Future work should address these effects.
Surface reflections, for instance, could be mitigated during image capture using polarizing filters. 
Light attenuation could be tackled by incorporating recent advances, such as WaterSplatting \parencite{li20243DV_watersplatting} and SeaSplat \parencite{Yang2025ICRA_SeaSplat}, which explicitly model and estimate these underwater characteristics. 
Furthermore, incorporating the Fresnel effect, which governs the view-dependent interplay between reflection and refraction, is crucial for higher fidelity. 
This could be implemented by applying a radiance reduction to each Gaussian's color parameter $\bm{c}$ based on the ray's angle of incidence.

Moreover, our geometry extraction pipeline currently relies on handcrafted heuristic priors. Alternative Gaussian Splatting methods with a stronger emphasis on geometric fidelity \parencite{Huang2024SIGGRAPH_2DGS,Gu2024CVPR_SuGaR,fan2024_trimGS} could further improve geometric accuracy. Introducing regularization terms that encourage Gaussians to align with surfaces could enable more faithful reconstructions of bed topography. Such regularization could also mitigate wave-induced distortions by prioritizing geometric plausibility over per-pixel fidelity.

Overall, our method represents an important step toward tackling refraction, one of the key challenges in photogrammetric bathymetry, and opens a new avenue for accurate shallow-water depth estimation.


% 自分で書いた初稿
% This study primarily focuses on refractive effects, and thus the method is currently applicable only to idealized refractive environments. 
% In real scenarios, reflections at the water surface and absorption within the water medium cannot be neglected, since our approach seeks to minimize radiometric difference to  observed images, which inherently capture reflection and attenuation. 
% Even with respect to the refractive correction, Fresnel's effect is not currently taken into account. 
% To do so, a reduction of radiance according to ray angle should be applied to each Gaussian color parameter $\bm{c}$.
% Polarizing filters could also mitigate surface reflections during image capture. 
% About absorption, which caused by spectral-depended
% absorption and backscattering of ambient light, should affect result.
% WaterSplatting \parencite{li20243DV_watersplatting} or SeaSplat \parencite{Yang2025ICRA_SeaSplat}, which model attenuation of light underwater and estimate, can be applied to scenarios where observations from UAVs.
% Similarly, absorption and scattering caused by wavelength-dependent attenuation of light underwater should be considered. 
% Recent approaches such as WaterSplatting \parencite{li20243DV_watersplatting} and SeaSplat \parencite{Yang2025ICRA_SeaSplat}, which explicitly model underwater light attenuation and are able to estimate these parameters, could potentially be extended to scenarios where observations from UAVs, not limited in underwater. 

% Moreover, our geometry extraction pipeline currently relies on handcrafted heuristic priors.
% Alternative related work of Gaussian Splatting with a stronger emphasis on geometric fidelity \parencite{Huang2024SIGGRAPH_2DGS,Gu2024CVPR_SuGaR,fan2024_trimGS} may help to improve geometric accuracy.
% Introducing regularization terms that encourage Gaussians to align along surfaces could enable sharper and more faithful reconstructions of bed topography.
% It could also mitigate wave-induced distortions by prioritizing geometric plausibility over per-pixel fidelity.

% Overall, our method represents an important step toward tackling refraction, one of the key challenges in photogrammetric bathymetry, exploring to a new way for accurate shallow-water depth estimation.

% --- 参考文献リストを出力 ---
\clearpage
\printbibliography


% ---- APPENDIX ----
\clearpage
\section{APPENDIX}
% これを実装する
\subsection{\sout{3D Covariance correction}}\label{covariance-correction}
Futere Work的 論文には載せないです。

memo 
overestimateするから、ローパスフィルタ的な感じで、Gaussianが萎みすぎないようにする?


By the transformation of Gaussians mean $\bm{p} \to \bm{p'}$ that equals to refractive scene effect, the Cartesian space of camera coordinates in the real 3D space is going to be distorted and compressed.
We term this space \textit{the appearance space} and this is no longer orthogonal like \cref{fig:space_compression_by_view}.

Particularly, when the ray of incident angle is larger, the apparent space is compressed significantly.
Without the correction about a scale $\bm{s}$ and a rotation $\bm{R}$, each Gaussian is observed to be dense and dilated.
This causes visual artifacts, such as blurring and unrealistic expansion of surfaces as shown in \cref{fig:artifact-w/o-scale-correction}.


So corrections about scales $\bm{s}$ and rotation $\bm{R}$ are needed to scale down Gaussians according to the compressed space of the apparent space.
Same as forward rendering process of 3DGS \cref{eq:3dgs-affine-projection}, we can express transformed 3D covariance $\Sigma^{3D}_{app}$ which include the scales and rotation as below, assuming each Gaussians are small enough and can be approximated by affine transformation.
\begin{equation}
  \Sigma^{3D}_{app} = J_{app} \Sigma^{3D}_{cam} J_{app}^\top
\end{equation}
Then, as in \cref{eq:3dgs-affine-projection,eq:3dgs-alpha-blending}, the rendering is done by perspective projection and alpha blending.


\subsection{\sout{Color correction}}\label{sec:color-correction}
同様に、Future Work。実装が間に合っていません。

Observing a 3D scene through a refractive interface, such as from air to water, results in a reduction of its apparent brightness. 
This phenomenon occurs even without considering Fresnel reflections at the boundary \parencite{frestnel_law} or attenuation from scattering and absorption. 
This is a direct consequence of the fundamental change in radiance governed by the law of etendue conservation. 

From Snell's Law, multiplying \cref{eq:snell,eq:dtheta_r/dtheta_i} yields:
\begin{equation}
  n^2 \cos \theta_i \sin \theta_i d \theta_i = n^2 \cos \theta_r \sin \theta_r d \theta_r
\end{equation}
Multiplying by a differential surface area  $dA$  and the azimuthal angle differential $d \psi$, we relate this to the differential solid angles $d\omega = \sin\psi d\psi d\theta$ \parencite{solid_angle}:
\begin{equation}\label{eq:etendue_preservation}
  \cos\theta_i d\omega_i dA = n^2 \cos\theta_r d\omega_r dA
\end{equation}
This relationship is known as the law of etendue conservation \parencite{etendue}. 
Etendue, a measure of how spread out light is in terms of both area and solid angle, is conserved across a lossless interface. 
Assuming the differential area dA remains constant for incident and refracted rays, the radiance must change. 
Specifically, for a light ray bundle traveling from water (radiance $L_w$) to air (radiance $L_a$), their relationship is:
\begin{align}\label{eq:radiance_between_air_and_water}
  L_a &= \frac{d^2 \Phi }{dA \cos\theta_i d\theta_i} \notag \\
      &= \frac{d^2 \Phi }{n^2 dA \cos\theta_r d\theta_r} \notag \\
      &= \frac{1}{n^2} L_w 
\end{align}
Thus, Radiance in water into air become reduce devided by $n^2$, and the brightness is lower and looks darker.
Thus, the radiance is reduced by a factor of $n^2$ when light passes from water to air, causing the scene to appear darker, as illustrated in Fig. \ref{fig:refraction-makes-scene-darker}.

\begin{figure}[htbp]
  \centering
  % 1枚目の画像
  \begin{minipage}[b]{0.45\linewidth}
    \centering
    \includegraphics[width=\linewidth]{figure/method/color/0090_gt_background.png}
  \end{minipage}
  \hfill % 画像間のスペースを最大にする
  % 2枚目の画像
  \begin{minipage}[b]{0.45\linewidth}
    \centering
    \includegraphics[width=\linewidth]{figure/method/color/0090_refraction_background.png}
  \end{minipage}
  \caption{A scene Rendered using a path tracer in Blender Cycles. The right image, viewed through a refractive water surface, is significantly darker than the left image without refraction.}
  \label{fig:refraction-makes-scene-darker}
\end{figure}

To compensate for this radiometric distortion, we introduce a correction that scales the intensity of each pixel $\mathrm{intensity}(\bm{\Gamma}(\bm{x}))$, devided by $n^2$.
In 3DGS, the radiance of the 3D scene is baked into the final RGB color values. 
Assuming the pixel intensity is determined after gamma correction:
\begin{align}
  \mathrm{intensity}(\Gamma(\bm{x})) = \langle\mathbf{I},\bm{\Gamma}^\gamma\rangle
\end{align}
where $ \mathbf{I}^\top = (0.2126, 0.7152, 0.0722)$  is the standard vector for converting sRGB to luminance \parencite{sRGB},
and we assume $\gamma=2.2$.

The target refractive intensity for the corrected scene $\Gamma'(\bm{x})$ should be:
\begin{align}
  \mathrm{intensity}(\Gamma'(\bm{x})) &= \frac{1}{n^2}\langle\mathbf{I},\bm{\Gamma}^\gamma\rangle \notag \\
                                      &= \langle\mathbf{I}, \frac{1}{n^2} \bm{\Gamma}^\gamma\rangle  \notag \\
\end{align}
This implies that the following condition must be satisfied:
\begin{align}
  & \left(\bm{\Gamma}'\right)^\gamma = \frac{1}{n^2} \bm{\Gamma}^\gamma \notag \\
  & \bm{\Gamma}' = n^{-\frac{2}{\gamma}} \Gamma
\end{align}
This correction can be achieved by scaling the color $\bm{c}'$ of each Gaussian by $n^{-\frac{2}{\gamma}}$.
The validity of this approach is demonstrated by substituting the corrected color $\bm{c}' = n^{-\frac{2}{\gamma}} \bm{c}$  into the volumetric rendering equation \cref{eq:3dgs-alpha-blending}:
\begin{align}\label{eq:correc-color-intensity}
  \bm{\Gamma}'(\bm{x}) &= \sum_{k=1}^{K} \bm{c}'_k \alpha^{\text{pixel}}_k \prod_{j=1}^{k-1} (1-\alpha^{\text{pixel}}_j) \notag \\
                      &= \sum_{k=1}^{K} \left(n^{-\frac{2}{\gamma}}\bm{c}_k\right) \alpha^{\text{pixel}}_k \prod_{j=1}^{k-1} (1-\alpha^{\text{pixel}}_j) \notag \\
                      &= n^{-\frac{2}{\gamma}} \sum_{k=1}^{K} \bm{c}_k \alpha^{\text{pixel}}_k \prod_{j=1}^{k-1} (1-\alpha^{\text{pixel}}_j) \notag \\
                      &= n^{-\frac{2}{\gamma}} \bm{\Gamma}(\bm{x})
\end{align}
For simplicity, our derivation considers only the base color of the Gaussians, neglecting view-dependent effects modeled by Spherical Harmonics.
While this formulation provides a physically-grounded correction for the base color, it does not account for view-dependent effects. 
Extending this model to appropriately scale the Spherical Harmonic coefficients remains a non-trivial challenge for future work, as the correction factor would depend on the outgoing light direction.


\subsection{詳細なApparent Depthの導出}

Here we assume the cartesian coordinates, which is like camera position is in \textit{z-axis} $(0, 0, H)$, and water surface is correspond to \textit{xy-plane}, flat and calm so it means the normal is constant like other research. % [TODO] 既存研究 
Now position $\bm{p}$ a 3D Gaussian primitive underwater water is represented as $(x, y, z) \text{ where } z < 0$.
To make it easy the problem to think, the position projects to 2D coordinates by $r = \sqrt{x^2 + y^2}$, into \textit{rz-plane}. % insert figure
Although this doesn't be well known, points underwater is determined at specific position from the view point position, which is not dependent for the view direction from Snell's law and geometric optics.
The appearance position of the Gaussian center is set as $\bm{p' (r', z')}$.
We lead that easily below.
The intersection between camera and appearance position can be set $(s, 0) \text{ where } 0 <  s < r $.
Snell's Law lead to below equation, by the set of the incident angle $\theta_i$, refractive angle $\theta_r$, refractive index of water $n$, and we assume the refractive index of air is 1. % incident angle という言い方は良くない。実際のrayの向きはtransmitされたものをみている。observed という。添字もiやrとしない
\begin{equation}\label{eq:snell}
  \sin \theta_i = n \sin \theta_r
\end{equation}
Each sine function can be represented as the ratio of triangle camera, the intersection and origin, or the intersection, the appearance position and the vertival line from the appearance and $\textit{r-axix}$.
% In \ref{}
$\triangle OAI$ and $\triangle IPC$
This leads to a quartic equation and the value of s can be got by solving this. % insert reference (refractive-aware SfM)
\begin{equation}\label{eq:quartic-s}
\begin{split}
  (1-n^2)s^4 &+ 2(n^2-1)rs^3 \\
             &+ ((1-n^2)r^2 + h^2 - n^2 H^2)s^2 \\
             &+ 2n^2H^2rs - n^2H^2r^2 = 0 % [TODO] have to check
\end{split}
\end{equation}
% in the implementation, i solved this by newton's method.
Here we think of the ray from the real Gaussian center to camera with small difference, the intersection defines as $(s + \Delta s, 0) \text{ where }  \Delta s < 0$, the incident angle is difined as $\theta_r + \Delta \theta_r$.
From the relation of $\triangle IP'C'$, 
\begin{equation}\label{eq:triangle-IP'C'}
  -z' \tan \theta_i  = r' - s 
\end{equation}
and from the relation of $\triangle I'P'C'$, 
\begin{align}
  -z' \tan \left(\theta_i + \Delta \theta_i \right) & = (r' - s) + (- \Delta s) \notag \\
  -z' \left( \sin \theta_i + \Delta \theta_i \cos \theta_i  \right) & = (r' -s - \Delta s) \left( \cos \theta_i - \Delta \theta_i \sin \theta_i \right) \label{eq:triangle-I'P'C'}
\end{align}

% By removing $z'$ from \cref{eq:triangle-IP'C', eq:triangle-I'P'C'}, we get
By removing $z'$ from \eqref{eq:triangle-IP'C'}, \eqref{eq:triangle-I'P'C'} and $\triangle \to 0$, we get
\begin{equation}\label{eq:limit-r'}
  r' = s - \sin \theta_i \cos \theta_i \frac{d s}{d \theta_i}
\end{equation}

With the equations of \cref{eq:limit-r',eq:triangle-IP'C'}, we get
\begin{equation}\label{eq:z'_ds/dtheta_i}
  z' = \frac{d s}{d \theta_i} {\cos \theta_i}^2
\end{equation}

From the triangle $\triangle IPC$, we get
\begin{equation}\label{eq:r-s}
  r - s = - z \tan \theta_r
\end{equation}

and derivalate by $\theta_r$, we get
\begin{equation}\label{eq:ds/dtheta_r}
  \frac{d s}{d \theta_r} = z \cdot \frac{1}{{\cos \theta_r}^2}
\end{equation}

With derivalating \cref{eq:snell} by $\theta_i$, we get
\begin{equation}\label{eq:dtheta_r/dtheta_i}
  \frac{d \theta_r}{d \theta_i} = \frac{1}{n} \cdot \frac{\cos \theta_i}{\cos \theta_r}
\end{equation}

With \cref{eq:dtheta_r/dtheta_i,eq:ds/dtheta_r}, we get
\begin{equation}\label{eq:ds/dtheta_i}
  \frac{d s}{d \theta_i} = \frac{z}{n} \cdot \frac{\cos \theta_i}{{\cos \theta_r}^3}
\end{equation}

With \cref{eq:limit-r',eq:z'_ds/dtheta_i,eq:ds/dtheta_i,eq:r-s}, we get
\begin{subequations}
  \begin{empheq}[left=\empheqlbrace]{align}
      r' &= r + (n^2 -1) \cdot z \cdot {\tan \theta_r}^3 \label{eq:r'} \\
      z' &= \frac{1}{n} \cdot \frac{{\cos \theta_i}^3}{{\cos \theta_r}^3} \cdot z \label{eq:z'}
  \end{empheq}
\end{subequations}

This is the center position correction transformation.
$\theta_i$ and $\theta_r$ are determined by the camera position and means of Gaussians. 
Therefore, the apparent Gaussian position relative to the view is uniquely determined, and we can correct means of Gaussians into apparent depth position $\bm{p'}(x', y', z')$.

$(n^2 -1) \cdot z \cdot {\tan \theta_r}^3 < 0$ therefore $r' < r$, and $\frac{1}{n} \cdot \frac{{\cos \theta_i}^3}{{\cos \theta_r}^3} < 1$ therefore $\left| z' \right| < \left| z \right|$, which ensures that the apparent Gaussian $G'(\bm{x})$ is always looks closer than the actual Gaussian $G(\bm{x})$. 
When $\theta_i = 0$, $r' = r$ and $z' = \frac{z}{n}$, which means that the apparent Gaussian is observed in the location which depth is devided by refractive index $n$.
Based on this situation, a lot of work like \parencite{nassar1994_ApparentDepth,Missailidis2025_apparentDepth-leading}, however, this is not exacly approximation in the majority of the cases because the camera rays with perspective range is off-nadir unless camera set down to vertical.

% Implementation details に記述するべき?
To validate the differentiableness in \textit{Gaussian Splatting}, we have to calculate the $\textit{Jacobian}$ of this transformation.

\begin{equation}\label{eq:Jacobian}
  \begin{split}
    J_{app} = 
    \frac{\partial \bm{p'}}{\partial \bm{p}} &= 
    \begin{bmatrix} 
      \frac{\partial x'}{\partial x} & \frac{\partial x'}{\partial y} & \frac{\partial x'}{\partial z} \\
      \frac{\partial y'}{\partial x} & \frac{\partial y'}{\partial y} & \frac{\partial y'}{\partial z} \\ 
      \frac{\partial z'}{\partial x} & \frac{\partial z'}{\partial y} & \frac{\partial z'}{\partial z} 
    \end{bmatrix} \\
  \end{split}
\end{equation}

This can be calculated analytically, here, we calculate it by numercial derivativation in the implementation. 
We set $\bm{p} \pm {(\Delta x, \Delta y, \Delta z)}^T$ and calculate the apparent postion by \cref{eq:r',eq:z'}, so for examle, the ${\partial x'} / {\partial x} = \lbrace p'(x+\Delta x) - p'(x-\Delta x) \rbrace / 2\Delta x$.

This correction is the key proposal in this paper, which modelize the effect of refraction between tha air and water, by simple position transformation. 
This correction enables estimate the accurate 3D scene without refraction from the posed image correction with refractive water surfece, unless we need to give the refractive index and location of the water surface relative to the cameras.
% [TODO] ↑後に Limitationで、水面位置や、nを微分可能な変数として考えることで、PyTorchの自動微分などを使用することで、画像スタックから推定可能と述べる。




\end{document}

