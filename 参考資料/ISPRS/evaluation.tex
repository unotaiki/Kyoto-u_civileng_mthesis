\section{Results}\label{sec:EVALUATION}

The evaluation is performed on the synthetic dataset described in \cref{sec:dataset}, where all training images contain distortions induced by a planar air–water interface. 
Our goal is to assess how effectively the proposed refractive corrections recover both appearance and geometry relative to ground truth.

\subsection{Appearance Evaulation}\label{sec:appearance-evaluation}


\begin{table*}[!tb]
  \caption{
    Quantitative appearance evaluation under two rendering conditions: (1) rendering with the refractive interface and comparing to refractive ground truth, and (2) rendering without the interface and comparing to refraction-free ground truth. 
    We show the ground truth, the baseline 3DGS, and ablation results where each component (Scale or Opacity) is added on top of the position correction.
    The full method (Ours) applies position, scale, and opacity correction simultaneously.
    All models are trained from refracted images only.}
  \label{tab:apparent-evaluation}
  \centering
  \small
  \scalebox{0.9}
  {
    \begin{tabular}{l||ccc|ccc|ccc|cc}
      
      & \multicolumn{3}{c|}{Correction Components} & \multicolumn{3}{c|}{Render w/ Refraction} & \multicolumn{3}{c|}{Render w/o Refraction} & \multicolumn{2}{c}{Stats}\\
      Method & Position & Scale & Opacity
      & $SSIM^\uparrow$   & $PSNR^\uparrow$    & $LPIPS^\downarrow$  
      & $SSIM^\uparrow$   & $PSNR^\uparrow$    & $LPIPS^\downarrow$  
      & \# Gaussians (K) & Time (min) \\
      \hline \hline 
      
      3DGS
      & \XSolidBrush & \XSolidBrush & \XSolidBrush
      % & 0.610 & 14.81 & 0.332 & 0.682 & 14.12 & 0.292 & 40.0 & 6.8 \\ 
      & - & - & - & 0.682 & 14.12 & 0.292 & 40.0 & 6.8 \\ 
      
      Position
      & \CheckmarkBold & \XSolidBrush & \XSolidBrush
      & 0.980 & 37.56 & 0.023 & 0.953 & 21.49 & 0.039 & 23.1 & 10.9 \\
      
      + Scale
      & \CheckmarkBold & \CheckmarkBold & \XSolidBrush
      & 0.964 & 37.21 & 0.023 & 0.936 & 22.97 & 0.043 & 28.0 & 21.0 \\

      + Opacity 
      & \CheckmarkBold & \XSolidBrush & \CheckmarkBold
      & 0.984 & 38.44 & 0.018 & 0.954 & 25.27 & 0.033 & 38.8 & 11.5 \\

      Ours
      & \CheckmarkBold & \CheckmarkBold & \CheckmarkBold
      & 0.981 & 38.42 & 0.017 & 0.933 & 25.97 & 0.046 & 41.3 & 21.9 \\

    \end{tabular}
  }
\end{table*}



Following established protocol in Novel View Synthesis (NVS), we evaluate similarity between rendered images from unseen viewpoints and ground truth.
We employ three widely-used metrics: the Peak Signal-to-Noise Ratio (PSNR), which measures pixel-wise accuracy; the Structural Similarity Index Measure (SSIM) \cite{Zhou2004_SSIM}, which considers local image structures; and the Learned Perceptual Image Patch Similarity (LPIPS) \cite{Zhang2018CVPR_LPIPS}, which leverages deep features from a Convolutional Neural Network (CNN) to better approximate human perception.
To fully characterize the impact of refraction, metrics are reported under two conditions (\cref{tab:apparent-evaluation}):
(1) rendering with the refractive interface and comparing to refracted ground truth, and
(2) rendering without the interface and comparing to refraction-free ground truth.

Across all metrics, Gaussian center position correction provides the largest performance gain, consistent with its role in resolving refractive ray-path deviations. 
SSIM, which is particularly relevant for downstream tasks such as riverbed classification, rises from 0.68 (baseline) to above 0.95 with position correction alone.

Opacity correction further improves perceptual quality, achieving the lowest LPIPS among all ablations. 
However, when evaluated in the non-refractive rendering condition, the PSNR remains considerably lower—approximately 13 dB below the refractive case. 
This is expected: as discussed in \cref{sec:opacity-correction}, the uniform opacity reduction preserves background visibility but does not physically model the reduction in radiance occurring at the refractive interface. 
Consequently, small luminance deviations are heavily penalized by PSNR despite being visually negligible.

Scale correction exhibits a mixed effect. 
Although geometrically beneficial (as shown in \cref{tab:geometry-evaluation}), isotropic covariance correction introduces slight blurring in fine textures, marginally reducing SSIM in non-refractive rendering. 
Nonetheless, all three corrections together yield a well-balanced representation with competitive perceptual fidelity (\cref{fig:ablation-appearance}).

Despite a training time roughly three times that of standard 3DGS, the absolute runtime remains practical, and can be further reduced via a dedicated CUDA implementation.


\subsection{Geometry Evalation}\label{sec:geometry-evaluation}

\begin{table*}[!tb]
  \caption{Quantitative results of Geometry evaluation using Chamfer Distance (CD) and F1 score at 10 cm and 30 cm thresholds. 
  All models are trained on refracted images.}
  \label{tab:geometry-evaluation}
  \centering
  \setlength\textfloatsep{0pt}
  \setlength\intextsep{0pt}
  \small
  \scalebox{0.90}
  {
    \begin{tabular}{l||ccc|c|ccc|ccc}
      
      & \multicolumn{3}{c|}{Correction Components} & &  \multicolumn{3}{c|}{10cm} & \multicolumn{3}{c}{30cm}\\

      Method & Position & Scale & Opacity
      & $CD^\downarrow$ (m)  
      & $Precise^\uparrow$ (\%)   & $Recall^\uparrow$ (\%) & $F1^\uparrow$ (\%)
      & $Precise^\uparrow$ (\%)   & $Recall^\uparrow$ (\%) & $F1^\uparrow$ (\%) \\
      \hline \hline 
      
      3DGS
      & \XSolidBrush & \XSolidBrush & \XSolidBrush
      & 8.477 & 0.03  & 0.20  & 0.06  & 0.12  & 2.52   & 0.23  \\
      
      Position
      & \CheckmarkBold & \XSolidBrush & \XSolidBrush
      & 0.110 & 69.80 & 84.65 & 76.51 & 82.80 & 99.95  & 90.57 \\
      
      + Scale
      & \CheckmarkBold & \CheckmarkBold & \XSolidBrush
      & 0.033 & 81.10 & 91.06 & 85.79 & 93.03 & 99.97  & 96.37 \\

      + Opacity 
      & \CheckmarkBold & \XSolidBrush & \CheckmarkBold
      & 0.054 & 79.29 & 90.80 & 84.66 & 90.64 & 99.99  & 95.09 \\

      Ours
      & \CheckmarkBold & \CheckmarkBold & \CheckmarkBold
      & 0.011 & 91.56 & 96.58 & 94.00 & 98.26 & 100.00 & 99.12 \\


    \end{tabular}
  }
\end{table*}

\begin{figure*}[!tb]
  \centering
  % \setlength\textfloatsep{2pt}
  % \setlength\intextsep{2pt}
  \includegraphics[width=0.65\linewidth]{figure/evaluation/real-space/appearance_align.png}
  \caption{
    Qualitative comparison of appearance reconstruction.
    We show Ground Truth, the baseline 3DGS, and ablations of individual correction components.
    }
  \label{fig:ablation-appearance}
\end{figure*}

Accurate reconstruction of underwater terrain from aerial imagery is fundamentally challenging due to refractive ray bending.
We evaluate geometric accuracy using the Chamfer Distance and the F1 score, two standard metrics for point-cloud comparison~\cite{Seitz2006_Comparison-MVS}.
The Chamfer Distance captures the average nearest-neighbor deviation between reconstructed and ground truth surfaces, while the F1 score balances precision and recall under a specified tolerance.
Following practical requirements for high-resolution bathymetric mapping~\cite{Agrafiotis2020_refractiveCorrection}, we adopt 10 cm and 30 cm thresholds, which represent realistically stringent criteria for operational use.


As shown in \cref{tab:geometry-evaluation}, the baseline 3DGS completely fails in the refractive scenario, yielding a CD of 8.477 m and near-zero F1 scores. 
Introducing position correction alone resolves most of the refractive displacement, reducing the CD by two orders of magnitude to 0.110 m.

Scale correction further refines spatial consistency, bringing the CD down to 0.033 m and improving the 10 cm F1 score from 76.51\% to 85.79\%.
While opacity correction has a smaller geometric influence, it contributes modest improvements in both precision and recall.

The full mode (Ours) — combining all three corrections — achieves the best performance, with a CD of only 0.011 m and F1 scores of 94.00\% (10 cm) and 99.12\% (30 cm). 
Notably, these results are obtained under water depths up to 10 m, demonstrating strong robustness to severe refractive distortion.

Qualitative comparisons in \cref{fig:geometry-evaluation-quantitatively} confirm these findings: the corrected geometry closely matches the ground truth in position and shape, whereas the uncorrected reconstruction collapses into a shallow, noisy surface.


\begin{figure*}[!ht]
  \setlength\textfloatsep{0pt}
  \setlength\intextsep{0pt}
  \centering
  % 1枚目の画像
  \begin{minipage}[b]{0.45\linewidth}
    \centering
    \includegraphics[width=\linewidth]{figure/evaluation/geometry/gt.png}
  \end{minipage}
  % \hfill % 画像間のスペースを最大にする
  % 2枚目の画像
  \begin{minipage}[b]{0.45\linewidth}
    \centering
    \includegraphics[width=\linewidth]{figure/evaluation/geometry/wo.png}
  \end{minipage}
  \begin{minipage}[b]{0.45\linewidth}
    \centering
    \includegraphics[width=\linewidth]{figure/evaluation/geometry/w.png}
  \end{minipage}
  \caption{Qualitative geometry comparison.
  Top-left: The ground truth geometry rendered in simulator.
  Top-right: The extracted geometry without correction.
  Bottom-center: The extracted geometry with all correction.
  Each image is rendered from an identical viewpoint and under the same lighting conditions. 
  The ground truth and the corrected geometry are very similar in both position and shape. 
  In contrast, the uncorrected geometry appears completely noisy and is incorrectly positioned at a shallower depth.
  }
  \label{fig:geometry-evaluation-quantitatively}
\end{figure*}

