\subsection{Opacity Correction}\label{sec:opacity-correction}

Reconstructing scenes observed through a refractive interface introduces a characteristic radiometric attenuation.
As illustrated in \cref{fig:refraction-makes-scene-darker}, a scene viewed through water appears substantially darker due to etendue conservation \cite{etendue}. 
For a light bundle traveling from a medium with refractive index $n$ into air, the radiance is scaled as
\begin{equation}
L_{\mathrm{air}} = \frac{1}{n^{2}} L_{\mathrm{water}}
\end{equation}
This reduction in observed radiance poses a challenge for 3D Gaussian Splatting. 
When optimizing directly against the darker target images, the solution often collapses into a configuration with artificially low opacity.
Such semi-transparent Gaussians explain the low observed brightness but lead to geometrically ambiguous, blurry, and under-constrained reconstructions.

To counteract these phenomena, we introduce an opacity regularization inspired by the radiometric relationship above. 
Rather than allowing the optimizer to freely reduce opacity, we compensate the expected radiance loss by scaling the opacity of Gaussians located behind the refractive interface:
\begin{equation}
\alpha' = \frac{1}{n^{2}} \alpha
\end{equation}
This adjustment reduces the opacity budget available to each Gaussian, thereby discouraging trivial low-opacity solutions. 
As a result, the optimizer is driven to represent the scene using a denser and more spatially coherent set of primitives, improving both geometric and apparent fidelity.

It is important to emphasize that this strategy is not a physically exact model of refraction. 
Instead, it is a principled heuristic that stabilizes the optimization in refractive environments. 
Empirically, it yields consistently sharper geometry and avoids common artifacts such as floaters and oversmoothing.

\begin{figure}[htbp]
  \centering
  % 1枚目の画像
  \begin{minipage}[b]{0.40\linewidth}
    \centering
    \includegraphics[width=\linewidth]{figure/method/color/0050_gt_background.png}
  \end{minipage}
  % \hfill % 画像間のスペースを最大にする
  % 2枚目の画像
  \begin{minipage}[b]{0.40\linewidth}
    \centering
    \includegraphics[width=\linewidth]{figure/method/color/0050_refraction_background.png}
  \end{minipage}
  \caption{
    Comparison of path-traced renderings in Blender Cycles.
    The right image, viewed through a water surface, exhibits substantial radiance reduction compared to the left image rendered without refraction.
    Our opacity regularization mitigates the tendency of Gaussian Splatting to converge to low-opacity, blurry configurations under such conditions.}\label{fig:refraction-makes-scene-darker}
\end{figure}