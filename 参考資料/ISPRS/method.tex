\section{Methods}\label{METHODOLOGY}
\sloppy

\begin{figure*}[htbp]
  \centering
  \includegraphics[width=0.9\linewidth]{figure/method/position/overview.pdf}
  \caption{
    Overview of Refractive-Aware Gaussian Splatting.
    A 3D scene represented by Gaussian primitives is transformed into an
    \textit{apparent space} for each camera view by applying a differentiable refractive transformation consisting of position, scale, and opacity corrections. 
    This models light bending at the air–water interface.
    The transformed Gaussians are rendered and compared with the input images (which themselves contain refractive distortions). 
    The reconstruction loss is backpropagated to optimize the original Gaussians, enabling recovery of a refraction-free 3D model and photorealistic novel views.
  }
  \label{fig:overview}
\end{figure*}

Our goal is to reconstruct a refraction-free 3D representation from aerial images that are distorted by the air–water interface.
To achieve this, we incorporate a physically grounded refractive transformation into 3D Gaussian Splatting (3DGS).
For each view, every Gaussian is mapped from its true underwater parameters $(\bm{p},\ \bm{s},\ \alpha)$ to an apparent set $(\bm{p}',\ \bm{s}',\ \alpha')$ via a differentiable three-component transformation: Position correction (\cref{sec:center-correction}), Scale correction (\cref{sec:scale-correction}), Opacity correction (\cref{sec:opacity-correction}).


\subsection{Position Correction}\label{sec:center-correction}


We assume a camera located at $(0,0,H)$, a planar and calm water surface at $z=0$, and an underwater Gaussian center $\bm{p} = (x, y, z)$, where $z < 0$.
Because refraction is rotationally symmetric around the viewing direction, we reduce the geometry to the \textit{rz}-plane using the radial distance $r = \sqrt{x^2 + y^2}$. 

Let $s$ be the radial coordinate of the ray–interface intersection
($0\le s < r$).
Let $\theta_r$ and $\theta_i$ denote, respectively, the incidence and
refraction angles with respect to the interface normal.
With refractive index $n$ for water (air assumed to be $1$), Snell’s law reads:
\begin{equation}
    \sin\theta_{i}=n\sin\theta_{r}
    \label{eq:snell}
\end{equation}

By relating these angles to the geometry of the triangles formed by the camera,
the interface, and the underwater point, we obtain a quartic equation in $s$ (triangle similarity between $\triangle OAI$ and $\triangle IPC$):
\begin{equation}\label{eq:quartic-s}
\begin{split}
  (1-n^2)s^4 &+ 2(n^2-1)rs^3 \\
             &+ ((1-n^2)r^2 + H^2 - n^2 z^2)s^2 \\
             &+ 2n^2z^2rs - n^2z^2r^2 = 0.
\end{split}
\end{equation}
The physically valid root $s$ is computed numerically using Newton iterations or Ferrari’s closed-form solution.

Once $s$ is known, the geometric relations ($\triangle IPC$, $\triangle IP'C'$, and $\triangle I'P'C'$):
\begin{align}
  r - s &= -z\tan\theta_r  \notag \\
  r' - s &= z'\tan\theta_i \notag \\ 
  -z' \tan \left(\theta_i + \Delta \theta_i \right) & = (r' - s) + (- \Delta s) 
  \label{eq:triangle-relations}
\end{align}
combined with \cref{eq:snell} and the derivative of Snell’s law:
\begin{equation}
  \frac{d\theta_r}{d\theta_i}
  =\frac{1}{n}\frac{\cos\theta_i}{\cos\theta_r} 
  \label{eq:dtheta_r/dtheta_i}
\end{equation}
lead to closed-form expressions for the apparent position:
\begin{subequations}
\begin{empheq}[left=\empheqlbrace]{align}
    r' &= r + (n^{2}-1)\,z\,\tan^{3}\theta_{r}, \label{eq:r'}\\[2pt]
    z' &= \frac{1}{n}\left(\frac{\cos\theta_i}{\cos\theta_r}\right)^{3} z. \label{eq:z'}
\end{empheq}
\end{subequations}

Because $z<0$ and $\theta_i>\theta_r$, these relations guarantee
$r'<r$ and $|z'|<|z|$; i.e., underwater points appear both shallower and closer to the optical axis (\cref{fig:rz}).  
For $r=0$ (nadir view), the formula simplifies to the well-known approximation
$z' = z/n$.

To ensure differentiability within 3DGS, we compute the Jacobian:
\begin{equation}
  J_{\text{app}}=\frac{\partial\bm{p}'}{\partial\bm{p}}
  \in\mathbb{R}^{3\times3}
  \label{eq:Jacobian}
\end{equation}
via central finite differences, for example:
\begin{equation}
  \frac{\partial x'}{\partial x} \approx \frac{p'_x(x+\Delta x) - p'_x(x-\Delta x)}{2\Delta x}
\end{equation}
This provides stable gradients while avoiding cumbersome analytical
derivations.

This position correction—an analytical mapping of Gaussian means through a
refractive interface—is the central contribution of our method and enables the
reconstruction of refraction-free geometry using only the refractive index and
interface height.



\begin{figure}[htbp]
  \centering
  \setlength\textfloatsep{0pt}
  \setlength\intextsep{0pt}
  \includegraphics[width=0.9\linewidth]{figure/method/rz.png}
    \caption{
    Illustration of the apparent position $P^{\prime}$ versus the true position $P$ in the \textit{rz}-plane.
    Even though the actual object is located at the blue Gaussian position $P$, it appears shifted toward the red Gaussian position $P'$ when observed through the refractive interface.
    }
    \label{fig:rz}
\end{figure}


\subsection{Scale Correction}\label{sec:scale-correction}

\begin{figure}[htpb]
  \centering
  \includegraphics[width=0.75\linewidth]{figure/method/space_compression.png}
  \caption{Space compression from the camera view. 
  The space is distorted by the camera position.
  Each colored grid represents the corresponding compressed space of entire grid in camera coordinates (shown in black.)}
  \label{fig:space_compression_by_view}
\end{figure}

Because the transformation $\mathbf{p}\rightarrow\mathbf{p}'$ is nonlinear, it induces a distorted coordinate system in the apparent space (\cref{fig:space_compression_by_view}).  
Without compensatory scaling, the space becomes unrealistically compressed, causing Gaussians to be blurred and expanded on their surface (\cref{fig:artifact-w/o-scale-correction}).

\begin{figure}[htbp]
  \centering
    \includegraphics[width=\linewidth]{figure/method/artifact-wo-scale-correction.png}
    \caption{Artifact without scale correction. 
    Left: A rendered image of the 3DGS model trained on the ground truth images without refractive plane.
    Right: A rendered image of the same model using center position correction, but without scale correction.
    The scene in the area bounded by red is expanded, where the incidence angle of the ray is larger}\label{fig:artifact-w/o-scale-correction}
\end{figure}

To mitigate these artifacts, we quantify local anisotropic distortion by examining how the Jacobian $J_{\text{app}}$ transforms the canonical basis.  
For example, the transformed $x$-axis basis vector is given by the first column of $J_{\text{app}}$ \cref{eq:Jacobian}:
\begin{equation}
  \bm{e}_x'=
  J_{\text{app}}\bm{e}_x=
    \begin{bmatrix}
        \partial x'/\partial x \\
        \partial y'/\partial x \\
        \partial z'/\partial x
    \end{bmatrix}
\end{equation}
whose norm defines the directional scale factor:
\begin{equation}
  s_x=\|\bm{e}_x'\|
  % = \sqrt{\left(\frac{\partial x'}{\partial x}\right)^2 + \left(\frac{\partial y'}{\partial x}\right)^2 + \left(\frac{\partial z'}{\partial x}\right)^2}
\end{equation}
Analogous factors $s_y$ and $s_z$ are obtained for the remaining axes.


A natural option is to use $\lvert\det J_{\text{app}}\rvert$ as a volumetric
compression measure.
However, we found this approximation to overcompensate near large incidence angles, producing excessively small Gaussians.  
Instead, we use the geometric mean of the directional scale factors:
\begin{equation}
  V=s_x\,s_y\,s_z
  \label{eq:volume-compression-ratio}
\end{equation}
and define an isotropic correction factor:
\begin{equation}
  S = V^{1/3}
  \label{scale-correction-factor}
\end{equation}
The corrected scale is then:
\begin{equation}
  \bm{s}' = S\,\bm{s}
  \label{eq:scale-correction-edge}
\end{equation}
This isotropic strategy eliminates refractive expansion artifacts while
preserving stable training and preventing primitives from expanding and shrinking excessively.
